The surface code is one of the most promising quantum error-correcting codes due to its high threshold and its suitability for planar implementations with near neighbor interactions. A plethora of recent work has focused on making surface codes practical by improving the classical processing of the code while minimizing the impact on the threshold. Here, we build on top of the recently introduced union-find decoder, a decoder with a quasi-linear time complexity and a high threshold. The modified decoder, dubbed the union-find node-suspension decoder, outperforms the union-find decoder on any lattice size while retaining a quasi-linear worst-case time complexity. \david{Moreover, both for small lattice sizes and also for low error rates}, the new decoder performs near identically to the minimum-weight perfect-matching decoder.

%Quantum computing has the potential to transcend the information technology as we know it. Small scale quantum systems are already possible today, and the goal is to scale up these quantum architectures to build practical quantum devices. A major limiting factor in quantum computing is the accumulation of errors that may be caused by various sources. A solution is to encode the logical information in a larger amount of physical qubits to increase resilience, thereafter to decode the information with a decoder. A recently proposed decoder dubbed the Union-Find (UF) decoder is fast and almost linear in its worst-case time complexity, but has a reduced performance compared with the Minimum-Weight Perfect Matching (MWPM) decoder. We propose a modification of the UF decoder that aims to further decrease the weight of the correction operator, similarly to the MWPM decoder. The modified decoder, dubbed the Union-Find Node-Suspension (UFNS) decoder, manages to have an improved performance compared to the UF decoder on any lattice size. For small lattice sizes and also for low error rates, the UFNS decoder performs near identically to the MWPM decoder. We manage to maintain a quasi-linear worst-case time complexity of $\m{O}(n\log{n})$. 