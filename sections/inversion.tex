\subsection{Parity inversion}\label{sec:inversion}

An unfortunate effect of the Node-Suspension data structure, which we dub \textbf{parity inversion}, causes a decrease in the performance of the algorithm as the lattice size is increased (see Figure \ref{threshold_ufbb}). We will demonstrate this effect through the example in Figure \ref{fig5}\emph{a}. Consider three instances of the node-tree of Figure \ref{fig0}; $\nset_a, \nset_b, \nset_c$, positioned closely to each other on the lattice. Every node in all node-trees have radius $\nicefrac{1}{2}$, and the node suspension in $n_1$ in each node-tree is 2. This means that if $n_1$ is suspended for two growth iterations, such as in Figure \ref{fig1}, all nodes have the same PMW. However, in the current example, the node-trees $\nset_a, \nset_b, \nset_c$ merge after 1 iteration. The merged cluster is odd, thus we recalculate the node parities and delays to find that the parities in the partition of the node-tree containing the nodes of $\nset_b$ have been inverted, and the node suspensions in this partition have increased dramatically. If the next merging event occurs on the node with the increased node suspension, the matching weight may be larger compared to the original UF decoder, which defies the current goal. 

\Figure[t!](topskip=0pt, botskip=0pt, midskip=0pt){tikzfigs/tikz-figure5.pdf}{
    The node suspension values for nodes for 3 odd node-trees $\{\nset_1, \nset_2, \nset_3\}$ of 3 nodes that grow and join into a single node-tree. (a)
    Node suspensions are calculated by setting $k_{inv}=1$ in equation \eqref{eq:delayequation}. In step 1, the growth in the outer nodes of each of the three node-trees are prioritized and the node-trees merge. In step 2, recalculation of the joined node-tree isperformed. Parities within the sub-tree of $\nset_2$ are now inverted, and the suspension in these nodes have doubled. (b) Node suspensions are calculated by setting $k_{inv}=\nicefrac{1}{2}$. Now the increase in node suspensions after parity inversion is halved.\label{fig5}}

This defines a trade-off in the node suspension; a node must wait as many iterations as it is suspended to reach equal PMW in the node-tree, but after a parity-inversion the node suspension for previously prioritized nodes increase linearly with the number of iterations waited by the suspended nodes pre-inversion. As a compromise, we redefine the node suspension as \textbf{half} the number of growth iterations needed for all nodes in the node-tree to reach equal PMW. This can be done in Equation \eqref{eq:delayequation} by setting $k_{inv}=0.5$.

Nevertheless, as more parity inversions occur, the maximum node suspension in the node-tree increases, and it becomes more and more unlikely for a cluster to actually reach zero node suspension in all nodes. The number of parity inversions is directly related to the number of merging events, and thus the size of the lattice. The performance to improve the heuristic for minimum weight matchings thus decreases for larger lattices. 

