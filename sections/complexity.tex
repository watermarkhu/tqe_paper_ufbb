
\section{Complexity of Node-Suspension}\label{sec:complexity}

\Figure[htb](topskip=0pt, botskip=0pt, midskip=0pt){tikzfigs/tikz-figure6.pdf}{The fragmentation of node-tree $\pre{k-1}\nset^o$ into its ancester node-trees, where the prefix $k_1$ indicates the fragmentation step number. In the intermediate fragmentation $f_o$, $\pre{k-1}\nset^o$ is fragmented into an odd ancestor $\pre{k}\nset^o_0$ and even ancestor $\pre{k}\nset^e_{-1}$. Consequently in the intermediate fragmentation $f_e$, $\pre{k}\nset^e_{-1}$ is fragmented to odd ancestors $\pre{k}\nset^o_{1}$ and $\pre{k}\nset^o_{2}$. With $n_f=2$, each intermediate fragmentation is the opposite of a join operation between twe node-trees. Futhermore, note that if $\abs{\vset_0} = \abs{\vset_1} = \abs{\vset_2}$, the vertex-tree sizes related to node trees $\pre{k}\nset^o_0, \pre{k}\nset^o_1, \pre{k}\nset^o_2$, each fragmentation step is the opposite of a growth iteration. \label{fig6}}

In this section, we will find the worst-cast time complexity of the Union-Find Node-Suspension decoder. The addition cost of the original Union-Find decoder can be split in two parts: (A) the depth-first-searches (DFS's) related to the (re)calculation of the node parities and node delays in line \ref{algo:pdc}, and (B) the DFS related to the growth of a cluster in line \ref{algo:grow}. We dub these two parts the \textbf{suspension cost} and the \textbf{growth cost}, respectively. The \codefunc{Join} operation in lines \ref{algo:joina}-\ref{algo:joinb} only has a linear addition to the cost.

\subsection{Suspension cost}\label{sec:suscomplexity}

The cost of node suspension calculation is equal to the number of nodes traversed in the DFS's of the node parities and node delays. To find this number $N_{sus}(N)$ analytically, we utilize the rules of cluster growth and mergers.


\begin{enumerate}[label=\textbf{R\arabic*},ref=R\arabic*]
  \item Joins between the node-trees between odd and even clusters always retains the root node of the odd cluster. \label{rjoin}
  \item The DFS's of the node parity and delay calculation are always performed just before cluster growth in an even partition of the node-tree starting from the pointer saved at the root node. \label{rgrow}
  \item Weight growth states that clusters are grown in the order of sizes of their vertex-trees. \label{rweight}
\end{enumerate}
From rules \ref{rjoin} and \ref{rgrow}, we conclude that $N_{sus}$ is proportional to the sum of sizes of all even node-trees during all mergers of the growth process on the lattice. Note that if many clusters merge within the same growth iteration, only the last even cluster counts towards the cost, since \ref{rgrow} ensures that the calculation is not performed on intermediate even partitions. To find the worst-case time complexity, we maximize $N_{sus}$, which is proportional to the computation time. We take a top-down approach of \textbf{cluster fragmentation}; starting from a single cluster that maximally occupies the lattice at the end of growth, and move back in time to find its ancestor clusters and their sizes. The maximization of $N_{sus}$ is in the repetitiveness of the recalculation over some parition of the final node-tree. 

\begin{definition}\label{def:fragmentation}
  Let the \emph{fragmentation} of an odd cluster with node-tree $\pre{k-1}\nset^o$ split it into a set of its ancester node-trees. Here $k-1$ indicates the \emph{fragmentation step number}, where larger step number $k$ refers to an ancestor node-tree of smaller size. Let the fragmentation $f$ be the combination of \emph{intermediate fragmentations} (IF) $f_o$, which fragments an odd node-tree into an even ancestor and an odd ancestor
  \begin{equation}\label{eq:pfo}
    f_o(\{\pre{k-1}\nset^o\}) = \m{F}^o_k = \{\pre{k}\nset^e, \pre{k}\nset^o_0 \}, 
  \end{equation}
  and $f_e$, which fragments even node-trees into $n_f$ odd ancestors
  \begin{equation}\label{eq:pfe}
    f_e(\{\pre{k}\nset^e\}) = \m{F}^e_k=\{\pre{k}\nset^{o}_1,...,\pre{k}\nset^o_{n_f}\},
  \end{equation}
  such that a fragmentation is
  \begin{equation}\label{eq:fstep}
    f(\{\pre{k-1}\nset^o\}) = \m{F}_k = \{\pre{k}\nset^o_0,\pre{k}\nset^{o}_1,...,\pre{k}\nset^{o}_{n_f}\}.
  \end{equation}
\end{definition}

A fragmentation $f$ is only possible on an odd node-tree $\nset^o$ with $|{\nset^o}| \geq 3$. The fragmentation functions does not remove single-node node-trees from a set when applied, such that for an odd cluster of finite size, after $k_m$ fragmentation steps all odd ancestors in $\m{F}_{k_{m}}$ are single nodes. Using fragmentations, we can sum over the fragmentation steps of the maximally occupying cluster with node-tree $\pre{0}\nset^o$ to find the sum of even node-tree sizes
\begin{equation}\label{eq:npdc}
  N_{sus} = 2\sum_{k=1}^{k_{m}}{ \sum_{ \pre{k}\nset_j^e \in \m{F}^o_k(\pre{0}\nset^o) }{ \abs{\pre{k}\nset_j^e}} },
\end{equation}
for which we require (a) the number of fragmentation steps $k_m$, (b) the number of odd ancestors $n_f$ in Equation \eqref{eq:pfo} and (c) the ratios between the node-tree sizes of an odd node-tree and its ancestors
\begin{equation}\label{eq:ratio}
  R_i = \frac{\abs{\pre{k}\nset^o_i}}{\abs{\pre{k-1}\nset^o}}, \hspace{0.5cm} \sum_{i=0}^{n_f}{R_i} = 1.
\end{equation}

Normally, the number of nodes in a cluster is bounded by the number of vertices $\abs{\nset}\leq \abs{\vset}$. However, to find the parameters of Equation \eqref{eq:npdc}, we assume that vertex-trees do not increase in size  when grown, and thus only mergers between clusters result in larger vertex-trees. As a consequence, nodes are effectively not allowed to increase in radius, and the number of nodes in the cluster equals the number vertices $\abs{\nset}=\abs{\vset}$. While this is not physically possible, the assumption maximizes the number of nodes on the lattice such that the sum in Equation \eqref{eq:pfo} forms an upper bound to $N_{sus}$.

\begin{lemma}\label{lem:evenconstant}
  For constant $R_i$ and $\abs{\nset}=\abs{\vset}$, the sum of all even node-tree sizes in every step is constant. 
\end{lemma}
\begin{proof}
  For $k=1$, there is a single even ancestor $\pre{1}\nset^e$ of size 
  \begin{equation*}
    \abs{\pre{1}\nset^e} = (1 - R_0)\abs{\pre{0}\nset^o}.
  \end{equation*}
  For $k=2$, every odd node-tree in $\m{F}_1=\{\pre{1}\nset^o_0,...,\pre{1}\nset^{o}_{n_f}\}$ is fragmented by IF $f_o$ to an even ancestor $\pre{2}\nset^e_i$ for $i \in \{0,...,n_f \}$, such that 
  \begin{equation*}
    \sum_{i=0}^{n_f}{\abs{\pre{2}\nset^e_i}}  = \sum_{i=0}^{n_f}{R_i(1 - R_0)\abs{\pre{0}\nset^o}}= (1 - R_0)\abs{\pre{0}\nset^o}.
  \end{equation*}
  The same is true for all fragmentation steps $k$. 
\end{proof}

\begin{theorem}\label{the:fragnumber}
  Fragmentation number $n_f=2$ maximizes $N_{sus}$ in Equation \eqref{eq:npdc}.
\end{theorem}
\begin{proof}
  The sum of even node-tree simzes in every fragmentation step is constant per Lemma \ref{lem:evenconstant}. Thus, \eqref{eq:npdc} is maximized by a largest possible $k_m$.  As $n_f$ increases the number of odd node-trees in each IF $f_o$, the average size of these odd node-trees decreases. As more fragmentations are applied, the node-tree sizes in the fragmented set approach the single-node node-tree. An increases in $n_f$ decreases the number of fragmentation steps as $ k_m \propto \frac{1}{n_f}$.  Hence, $N_{PDC}$ is maximized for the minimal value of $n_f$, which is $n_f = 2$.
\end{proof}

Every fragmentation step on an odd cluster $\pre{k-1}\nset$ splits it into three parts $\{\pre{k}\nset^o_0, \pre{k}\nset^o_1, \pre{k}\nset^o_2\}$, where $\pre{k}\nset^o_1, \pre{k}\nset^o_2$ are ancestors of the even node-tree $\pre{k}\nset^e$ in the IF $f_o$. 

\begin{lemma}\label{lem:chrono}
  The node-tree size of $\pre{k}\nset_0$ must be smaller than $\pre{k}\nset^o_1, \pre{k}\nset^o_2$, such that $R_1 \geq R_0 \leq R_2$. 
\end{lemma}
\begin{proof}
  The intermediate fragmentations must occur in the order of \eqref{eq:pfo} and \eqref{eq:pfe}, as \eqref{eq:pfe} requires an even node-tree that is fragmented by \eqref{eq:pfo}. In terms of cluster growth, the vertex-trees $\vset_1, \vset_2$, corresponding to $\nset^o_1, \pre{k}\nset^o_2$, must merge before the combined vertex-tree can merge with $\vset_0$, which corresponds to $\pre{k}\nset_0$. Declared by rule \ref{rweight}, $\abs{\vset_1}$ and $\abs{\vset_2}$ must be smaller or equal to $\abs{\vset_0}$, such that 
  \begin{equation*}
    \abs{\vset_1}\geq \vset_0 \leq\abs{\vset_2}.
  \end{equation*}
  If this condition is not met, the cluster of $\vset_0$ grows first and merges with either $\vset_1$ or $\vset_2$, and the chronolonogy of events is disturbed. Since we assumed $\abs{\nset}=\abs{\vset}$, this can be translated to 
  \begin{equation*}
    \abs{\nset_1}\geq \nset_0 \leq\abs{\nset_2},
  \end{equation*}
  and to the ratio to the descendant node-tree of Equation \eqref{eq:ratio}.
\end{proof}

\begin{theorem}\label{the:ratios}
  The set of node-tree ratios that maximizes $N_{sus}$ in Equation \eqref{eq:npdc} is $R_0 = R_1 = R_2 = \nicefrac{1}{3}$.
\end{theorem}
\begin{proof}
  The ratios $\{R_0, R_1, R_2\}$ can be found by maximizing the size of the even node-tree $\pre{k}\nset^e$ in each fragmentation, which is 
  \begin{equation*}
    \abs{\pre{k}\nset^e} = (R_1 + R_2)\abs{\pre{k-1}\nset^o}.
  \end{equation*}
  Since $ R_1 \geq R_0 \leq R_2$ per lemma \ref{lem:chrono}, the largest values for $R_1, R_2$ possible are equal to $R_0$.
\end{proof}

The last unknown parameter for the maximization of $N_{sus}$ in Equation \eqref{eq:npdc} is $k_m$, the total number of fragmentation steps.

\begin{theorem}\label{the:km}
  For $n_f = 2$ and $R_i = \{\nicefrac{1}{3},\nicefrac{1}{3},\nicefrac{1}{3}\}$, the maximum number of fragmentation steps is $k_m = \log_3{\abs{\pre{0}\nset^o}}$.
\end{theorem}
\begin{proof}
  In every fragmentation step, all node-trees are fragmented into 3 ancestors that are $\nicefrac{1}{3}$ the size of their descendant. The series of $k_m$ fragmentations is thus simply $k_m$ divisions of the node-tree $\pre{0}\nset^o$ in 3 parts until all ancestors have size 1.
\end{proof}

Now, all variables that are needed to find the upper bound to $N_{sus}$ have been collected. From Lemma \ref{lem:evenconstant} and Theorem \ref{the:ratios}, we find that in every fragmented set $\m{F}^o_k$ the sum of even node-tree sizes is $\nicefrac{2}{3}\abs{\pre{0}\nset^o}$. Filling this and the value for $k_m$ in Equation \eqref{eq:npdc} we find that

\begin{align*}
  \nonumber N_{sus} &\leq 2\sum_{k=1}^{k_{m}}{ \sum_{ \pre{k}\nset_j^e \in \m{F}^o_k(\pre{0}\nset^o) }{ \abs{\pre{k}\nset_j^e} } } \\
  \nonumber         &\leq 2\sum_{k=1}^{\log_3{\abs{\pre{0}\nset^o}}} \frac{2}{3}\abs{\pre{0}\nset^o}\\
                    &\leq \frac{4}{3}\abs{\pre{0}\nset^o}\log_3{\abs{\pre{0}\nset^o}}.
\end{align*}

The maximum size of $\abs{\pre{0}\nset^o}$ is bounded by the system size $N = \abs{V}$. The worst-case time complexity of the suspension calculation is thus $\m{O}(N\log{N})$. 

\subsection{Growth cost}\label{sec:growthcost}

To grow a cluster represented by a node-tree $\nset$, a depth-first search (DFS) is performed on the node-tree to find all nodes that have zero suspension. The total cost of these DFS's are proportional to the total number of nodes encountered during these DFS's, which we dub $N_{grow}$. Using the definition of fragmentations of Definition \ref{def:fragmentation}, the cost of growth is propotional to
\begin{equation}\label{eq:ngrow}
  N_{grow} = 2\sum_{k=1}^{k_{m}}{ \sum_{ \pre{k}\nset_j^o \in \m{F}_k(\pre{0}\nset^o) }{ \abs{\pre{k}\nset_j^o} } }.
\end{equation}
We assume again that no trivial vertices are added  to a cluster or $|\nset| = |\vset|$. Determined by rule \ref{rgrow}, $N_{grow}$ is maximized if there are as many as possible mergers to a combined odd node-tree, while a minimum number of ancestor node-trees are involved. This is again achieved through $n_f = 2$. Within each fragmentation step, the DFS's are performed on each of the node-trees $\{\pre{k}\nset^o_0, \pre{k}\nset^o_1, \pre{k}\nset^o_2\}$ if all three clusters are grown. This is the case for $R_0 = R_1 = R_2$ such that $\abs{\vset_0}= \abs{\vset_1}=\abs{\vset_2}$, and all three clusters are grown simultanously (rule \ref{rweight}). 

For these values of $n_f$ and $R$, we can apply Theorem \ref{the:km} for $k_m$. For $|\nset| = |\vset|$, the sum of node-tree sizes in every fragmented set $\m{F}_k$ is exactly $\abs{\pre{0}\nset^o}$, and Equation \eqref{eq:ngrow} becomes an upper bound to $N_{grow}$: 
\begin{align*}
  \nonumber N_{grow} &\leq 2\sum_{k=1}^{k_{m}}{ \sum_{\pre{k}\nset_j^o \in \m{F}_k(\pre{0}\nset^o)}{ \abs{\pre{k}\nset_j^o} } } \\
  \nonumber         &\leq 2\sum_{k=1}^{\log_3{\abs{\pre{0}\nset^o}}} \abs{\pre{0}\nset^o}\\
                    &\leq 2\abs{\pre{0}\nset^o}\log_3{\abs{\pre{0}\nset^o}}.
\end{align*}
which again corresponds to a worst-case time complexity $\m{O}(N\log{N})$.