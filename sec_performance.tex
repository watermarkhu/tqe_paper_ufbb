\section{Performance}\label{sec:performance}

We benchmark the performance of our own variants of the Union-Find decoder, as well as the Union-Find Partitioned-Growth (UFPG) decoder of Algorithm \ref{algo:ufbb}. Decoding rate $d$ for a given lattice size $L$ and physical error rate is acquired by Monte Carlo simulations. We compare the performance under the independent noise model of only i.i.d. bit-flip errors with chance $p_X$ and the phenomenological noise model, which adds faulty syndrome measurements occurring at chance $p_X$. Code thresholds $p_{\text{th}}$ are obtained by curve fitting for the crossing point of decoding rates plotted in $(p_X, d)$ space, for a range of values for $L$ and $p_X$ \cite{wang2003confinement}. We also use the decoding rate at the threshold $d(p_{\text{th}})= d_{\text{th}}$ as a metric to compare decoders. 

\subsection{Union-Find decoder variants}

First, we show that $p_{\text{th}}$ can be increased, and the matching weight $\abs{\m{C}}$ can be decreased within the Union-Find decoder \textbf{without} the Partitioned-Growth data structure. We compare distinct variants of our implementation of the Union-Find decoder, that either implements Weighted Growth via bucket sort or no Weighted Growth, and either constructs a forest $F$ of grown clusters post-growth, which is the case in the original Union-Find decoder, or maintains acyclic vertex-trees $\vset$ during cluster growth. The latter is achieved by applying lines \ref{algo:dfa}-\ref{algo:dfb} of Algorithm \ref{algo:ufbb} in the UF decoder. The labels used for each variant are listed in \Cref{tab:uftable}. The full descriptions for each of the variants can be found in \cite{markthesis}.

\Figure[b](topskip=0pt, botskip=0pt, midskip=0pt){figures/python/comp_matching_weight.pdf}{
  The matching weight $\abs{\m{C}}$ of the Union-Find decoder variants (Table \ref{tab:uftable}) and the UFPG decoder, normalized to the minimum weight $\min{\abs{\m{C}}}$ of the Minimum-Weight Perfect Matching decoder. All weights are obtained by Monte Carlo simulations on $p_X=0.098$ with a minimum of $100.000$ samples. The x-axis scales linearly with $N = L^2$. \label{comp_weight}}

\begin{table}[htb]
  \centering
  \begin{tabularx}{\linewidth} { | R{2} || C{.5} | C{.5} | }
    \hline
    & $\mathbf{F}$ &  $\pmb{\vset}$\\
    \hhline{|=::=:=|}
    No Weighted Growth & fUF  & vUF \\
    \hline
    \textbf{B}ucket sort Weighted Growth & bfUF & bvUF \\
    \hline
  \end{tabularx}
  \caption{Abbreviated names for the variants of the Union-Find decoder.}\label{tab:uftable}
\end{table}


\Figure[bt](topskip=0pt, botskip=0pt, midskip=0pt){figures/python/threshold_ufpg.pdf}{
  The decoding rates $d$ of \emph{(a)} the UFPG decoder and \emph{(c)} the bvUF and MWPM decoders, obtained via Monte Carlo simulations with a minimum of 100.000 samples per lattice size per error rate. \emph{(a)} The decoding rates of the UFPG decoder do not cross in a single point, such that there is no apparent code threshold. \emph{(b)} The intersections of threshold curves of subsequent lattice sizes are the so-called threshold coordinates, which follow a trend in $(p_X, d)$ space as the lattice sizes increase. \emph{(c)} The threshold coordinates of the UFPG decoder occupy the region in $(p_X, d)$ space of the MWPM decoder. \label{threshold_ufpg}}
  
\begin{table}[htb]
  \centering
  \begin{tabularx}{\linewidth} { | R{1} || C{1} | C{1} | C{1} | C{1} | }
    \hline
    \multirow{2}{*}{} & \multicolumn{2}{c|}{Independent}& \multicolumn{2}{c|}{Phenomenological} \\
    \cline{2-5}
      & $p_{\text{th}}$ & $d_{\text{th}}$ & $p_{\text{th}}$ & $d_{\text{th}}$ \\
    \hhline{|=::=:=:=:=|}
    fUF & $9.72\%$ & $73.34\%$ & $ 2.53\%$ & $92.39\%$ \\
    \hline
    vUF & $9.79\%$ & $74.32\%$ & $2.56\%$ & $93.64\%$ \\
    \hline
    bfUF & $9.98\%$ & $72.71\%$ & $2.68\%$ & $91.32\%$ \\
    \hline
    bvUF & $10.01\%$ & $72.86\%$ & $2.69\%$ & $92.08\%$ \\
    \hline
    MWPM & $10.35\%$ & $71.58\%$ & $2.97\%$ & $90.24\%$\\
    \hline
  \end{tabularx}
  \caption{Threshold error rates $p_{\text{th}}$ and threshold decoding success rates $d_{\text{th}}$ for the implementations of the  Union-Find decoder of \Cref{tab:uftable}.}\label{tab:ufndfwug}
\end{table}


\Figure[t](topskip=0pt, botskip=0pt, midskip=0pt){figures/python/comp_time.pdf}{
  The mean computation time of the UFPG, bvUF, and MWPM decoders near the threshold error rate. All weights are obtained by Monte Carlo simulations for $p_X=0.098$ with a minimum of $100.000$ samples. The x-axis scales linearly with $N = L^2$.\label{comp_time}}

The values for $p_{\text{th}}$ and $d_{\text{th}}$ for each variant, including the Minimum-Weight Perfect Matching (MWPM) decoder, are listed in \Cref{tab:ufndfwug}. Weighted Growth has the expected behavior of increasing $p_{\text{th}}$. While there is no major increase in $p_{\text{th}}$ from the $\vset$-variants over the $F$-variants, a significant increase in $d_{\text{th}}$ can be observed. We suspect that the acyclic graphs of $\vset$ have shorter branches in between junctions compared to $F$, which leads to a decreased matching weight and increased $d_{\text{th}}$. We plot the matching weight $\abs{\m{C}}$ of the UF variants, normalized to the minimum weight of the MWPM decoder for $p_X = 0.098$ in \Cref{comp_weight}. Here we see a correlation between a decrease in $\abs{\m{C}}$ and increase in performance: both Weighted Growth and maintaining $\vset$ during growth increase performance and decrease matching weight. Furthermore, the matching weight in $\vset$-variants have a relatively low and constant factor over the minimum weight, which improves upon the $L$-dependent behavior of $F$-variants.


\subsection{Union-Find Partitioned-Growth decoder}

We benchmark the performance of the Union-Find Partitioned-Growth (UFPG) decoder of Algorithm \ref{algo:ufbb}. The decoding rates are plotted in \Cref{threshold_ufpg}\emph{a} per lattice size. We discover that the curves related to the decoding rates do not cross in a single point, such that there is no clear threshold $p_{\text{th}}$. In fact, the intersections of two curves of subsequent lattice sizes, which we dub threshold coordinates, follow a trend where larger input lattice sizes results in a decrease in $p_{\text{th}}$ but an increase in $d_{\text{th}}$. We ascribe the degradation of the threshold coordinate for larger lattices to the Parity Inversion effect. These threshold coordinates $(p_{\text{th}}, d_{\text{th}})$ are plotted in \Cref{threshold_ufpg}\emph{b}. When these coordinates are plotted together with the bvUF and MWPM decoders' decoding rates, such as in \Cref{threshold_ufpg}\emph{c}, we see that the threshold coordinate for small lattice sizes is similar to the MWPM threshold coordinate. For larger lattice sizes, UFPG threshold coordinates move towards the bvUF threshold on the $p_X$ axis, but still has an increased performance due to the increased $d_{\text{th}}$. Overall, the threshold coordinates occupy a region in $(p_X, d)$ space previously reserved to the MWPM decoder. 

The matching weight $\abs{\m{C}}$ of the UFPG decoder is successfully decreased compared to all UF variants of \Cref{tab:uftable}. For $p_X = 0.098$, an error rate close to $p_{\text{th}}$ for all decoders, the normalized matching weight is halved compared to the bvUF decoder (\Cref{comp_weight}). We compare the average running time of Monte Carlo simulations to obtain a matching between the UFPG, bvUF, and MWPM decoders. A comparison for a physical error rate near the threshold is included in \Cref{comp_time}. The average running time of UFPG, while not behaving according to worst-case $\m{O}(n \log{n})$, is also not linear. 

Finally, we show in \Cref{comp_lowerror} the performance of the UFPG decoder in the low-error regime with phenomenological noise. The decoding rate of the UFPG decoder is improved from the bvUF decoder and behaves similarly to the MWPM decoder. The mean computation time of the decoders in this regime, plotted in \Cref{comp_lowerror_time}, shows that the UFPG decoder performs at about the same speed as the bvUF decoder.\par

The simulator for the surface code, the Union-Find decoder variants, and the Union-Find Partitioned-Growth decoder have all been implemented in Python using our application \cite{qsurface}. The MWPM decoder utilizes the C implementation of BlossomV \cite{kolmogorov2009blossom} due to substantial slow performances of Python implementations. Simulations were initially performed on a single 3.20 GHz Intel Core i5 CPU but later parallelized on all 24 threads of 3.60 GHz Intel Xeon E5 CPU's. 