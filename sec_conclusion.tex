\section{Conclusion}\label{sec:conclusion}

In this paper, we have introduced a modification of the Union-Find (UF) decoder \cite{delfosse2017almost} that selectively grows regions of clusters based on the concept of a potential matching weight. The modified decoder, dubbed the Union-Find Node-Suspension (UFNS) decoder, relies on an additional data structure to facilitate the calculation of the potential matching weight. We have proved analytically that the UFNS decoder has a worst-case time complexity of $\m{O}(n\log{n})$. 

The Union-Find Node-Suspension decoder proposed here, similarly to the original Union-Find, applies to any surface code of any genus, with or without boundary, and to color codes \cite{delfosse2017almost}. For simplicity, we have focused on the standard implementation of the surface code without boundary and leave additional numerical evaluation for future work.

Through Monte Carlo simulations on various decoder types, we have found that the UFNS decoder improves upon the performance of the UF decoder for all tested physical error rates and system sizes. Unfortunately, there is no fixed error threshold due to the Parity Inversion effect, which affects the performance at larger lattice sizes. Nevertheless, the UFNS decoder manages to occupy a region in $(p_X, d)$ space previously reserved to the Minimum Weight Perfect Matching (MWPM) decoder. For the low-error regime, the UFNS combines the advantages of the MWPM decoder's high decoding rates and the UF decoder's low computation time. Future work should focus on finding a way around the Parity Inversion effect and testing the decoder for other error types, such as erasure errors. 

Recent work that includes the Union-Find decoder focuses on bringing the decoder algorithm to the hardware level. Most notably, a scalable decoder micro-architecture has been proposed with a fully pipelined hardware implementation \cite{das2020scalable}. Related work has shown that a reduction in bandwidth is possible provided qubits with a low physical error rate \cite{delfosse2020hierarchical}. Furthermore, another variant of the decoder, dubbed the \emph{Weighted Union-Find} decoder, not to be confused with \emph{Weighted Growth}, promises to increase the code threshold under circuit-level noise \cite{huang2020fault}. This application relies on adopting the decoder to a \emph{weighted} graph. Every edge $e\in\m{E}$ may now have a different length value, and edges are not limited to the growth of half-edges per growth iteration. We believe that the Union-Find Node-Suspension decoder and the Weighted Union-Find decoder are compatible. In the combined decoder, boundary edges in every node are grown with respect to their weights in the weighted graph. 

The Union-Find decoder manages to decode fast and scale almost-linearly with the input system size. However, these speed-ups come at the cost of a decreased decoding performance. With the Union-Find Node-Suspension decoder, we manage to find a middle ground between the two objectives; high decoding performance that runs in worst-case quasilinear time. For these reasons, it may be a great candidate for physical applications in the near future.

\Figure[b!](topskip=0pt, botskip=0pt, midskip=0pt){figures/comp_lowerror.pdf}{
  The decoding rate $d$ for the low-error regime of phenomenological noise for the MWPM, UFNS and bvUF decoders. The UFNS decoding rates are improved from the UF variants and are very similar to MWPM. All $d$ are obtained by Monte Carlo simulations with a minimum of $100.000$ samples. The x-axis scales linearly with $N = L^3$.\label{comp_lowerror}}

\Figure[b!](topskip=0pt, botskip=0pt, midskip=0pt){figures/comp_lowerror_time.pdf}{
  The mean computation time of the UFNS, bvUF, and MWPM decoders in the low error regime for phenomenological noise for $p_X = \{0.5\%, 1.2\%, 2\%\}$ of the same simulation as in \Cref{comp_lowerror}. In this regime, the UFNS computation times are very comparable to the bvUF decoder. The x-axis scales linearly with $N = L^3$. \label{comp_lowerror_time}}


\Figure[htb](topskip=0pt, botskip=0pt, midskip=0pt){figures/threshold_comparison.pdf}{
  Direct comparison of the performance of various decoders covered in this thesis. The data of the original Union-Find (UF) decoder is taken from its publication \cite{delfosse2017almost}. Using the same range of lattice sizes and error rates, we simulate and plot the performance of \emph{(1)} our implementation of the Union-Find decoder with Weighted Growth applied via bucket sort and acyclic vertex-trees maintained during growth, the bvUF decoder, \emph{(2)} the Union-Find Node-Suspension decoder (UFNS), and  \emph{(3)} the Minimum-Weight Perfect Matching (MWPM) decoder.\label{thres_comp}}