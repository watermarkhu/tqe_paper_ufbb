\documentclass{ieeeaccess}
\usepackage[OT1]{fontenc} 
\usepackage{amsfonts, amsmath, amssymb, amsthm}
\usepackage{algorithmic}
\usepackage{graphicx}
\usepackage{caption}
\usepackage{placeins}
% \DeclareCaptionFont{ieeeblue}{\color{accessblue}}
% \DeclareCaptionLabelFormat{myformat}{\figcapfont{\textbf{#1}\textbf{#2}}}
% \captionsetup{labelfont={bf,ieeeblue},labelformat=myformat}
\usepackage{setspace}
\usepackage[font={sf,scriptsize,stretch=0.84}, labelfont={bf,color=accessblue}]{caption}

\usepackage{tabularx, hhline, multirow}
\newcolumntype{L}[1]{>{\hsize=#1\hsize\raggedright\arraybackslash}X}%
\newcolumntype{R}[1]{>{\hsize=#1\hsize\raggedleft\arraybackslash}X}%
\newcolumntype{C}[1]{>{\hsize=#1\hsize\centering\arraybackslash}X}%
\newcommand{\gc}{\cellcolor[gray]{0.9}}

\usepackage{enumitem}
\usepackage{textcomp}
\usepackage{nicefrac}
\usepackage{mathtools}
\DeclarePairedDelimiter{\abs}{\lvert}{\rvert}
% \def\BibTeX{{\rm B\kern-.05em{\sc i\kern-.025em b}\kern-.08em
%     T\kern-.1667em\lower.7ex\hbox{E}\kern-.125emX}}
\usepackage[style=ieee, sorting=none]{biblatex}
\addbibresource{cit.bib}

% \usepackage[table]{xcolor}
% % Define colors

\newtheorem{definition}{Definition}[section]
\newtheorem{lemma}{Lemma}[section]
\newtheorem{theorem}{Theorem}[section]
\newtheorem{proposition}{Proposition}[section]

% Set new commands
\newcommand{\codeword}[1]{\texttt{\textcolor{MidnightBlue}{#1}}}
%\newcommand{\codefunc}[1]{\texttt{\textcolor{OliveGreen}{#1}}}
\let\oldemptyset\emptyset
\let\emptyset\varnothing
\newcommand{\codefunc}[1]{\texttt{#1}}
\newcommand{\m}[1]{\mathcal{#1}}
\newcommand{\n}[1]{\mathscr{#1}}
\newcommand{\bound}{\mathscr{B}}
\newcommand{\akker}{\mathscr{A}}
\newcommand{\nset}{\mathcal{N}}
\newcommand{\vset}{\mathcal{V}}
\newcommand{\pre}[1]{ {}^{#1} }
\newcommand{\ceil}[1]{{\left \lceil #1 \right \rceil }}
\newcommand{\floor}[1]{{\left \lfloor #1 \right \rfloor }}

% Set algorithm2e package settings
\usepackage[linesnumbered, ruled, vlined]{algorithm2e}
\SetAlgoCaptionLayout{centerline}
\setlength{\algoheightrule}{1pt}
\setlength{\algotitleheightrule}{1pt}
\setlength{\interspacetitleboxruled}{.5em}
\SetStartEndCondition{ }{}{}
\SetKwProg{Fn}{def}{\string:}{}
\SetKw{KwTo}{in}
\SetKwFor{For}{for}{\string:}{}
\SetKwIF{If}{ElseIf}{Else}{if}{ then}{else if}{else}{}
\SetKwFor{While}{while}{ do}{}
\SetInd{0.1em}{0.5em}
\SetAlgoNoEnd\DontPrintSemicolon
\SetAlFnt{\small}


\begin{document}
\history{Date of publication xxxx 00, 0000, date of current version xxxx 00, 0000.}
\doi{10.1109/TQE.2020.DOI}

\title{Quasilinear Time Decoding Algorithm for \\Topological Codes with High Error Threshold}
\author{
    \uppercase{S. Hu}\authorrefmark{1},
    \uppercase{and D. Elkouss\authorrefmark{2}}
}
\address[1]{Department of Physics, Delft University of Technology (email: watermarkhu@outlook.com)}
\address[2]{QuTech, Delft University of Technology, Lorentzweg 1
2628CJ Delft, The Netherlands (email: d.elkousscoronas@qutech.nl)}

\tfootnote{This work was partially supported by the Netherlands Organization for Scientific Research (NWO/OCW), as part of the Quantum Software Consortium program (project number 024.003.037/3368). }

\markboth
{S. Hu \headeretal: IEEE Transactions on Quantum Engineering}
{S. Hu \headeretal: IEEE Transactions on Quantum Engineering}

\corresp{Corresponding author: First A. Author (email: author@ boulder.nist.gov).}

\begin{abstract}
    Quantum computing has the potential to transcend the information technology as we know it. Small scale quantum systems are already possible today, and the goal is to scale up these quantum architectures to build practical quantum devices. A major limiting factor in quantum computing is the accumulation of errors that may be caused by various sources. A solution is to encode the logical information in a larger amount of physical qubits to increase resilience and to decode the information with a decoder. 
    
    We introduce a modification of the Union-Find decoder that aims to further decrease the weight of the correction operator, similarly to the Minimum-Weight Perfect Matching decoder. The modified decoder, dubbed the Union-Find Balanced-Bloom decoder, manages to have an increased error threshold compared to the Union-Find decoder of any lattice size. For small lattice sizes, the Union-Find Balanced-Bloom decoder performs near the code threshold of the Minimum-Weight Matching decoder. We manage to maintain a quasilinear worst-case time complexity of $\m{O}(N\log{N})$. 
\end{abstract}

\begin{keywords}
    Quantum Computing, Quantum Error Correction, Surface Code
\end{keywords}

\titlepgskip=-15pt
\maketitle

\section{Introduction}\label{sec:introduction}
One of the most promising approaches for fault-tolerant quantum computation is based on surface quantum error correcting codes \cite{dennis2002topological, kitaev2003fault}. With surface codes, error correction only requires the measurement of local operators on a 2-dimensional lattice of qubits. The measurement outcome, called the syndrome, is passed to the decoding algorithm to deduct the error that has occurred and to supply a correction operator. The resilience against errors can be improved by increasing the system size whilst the physical error rate is below a threshold value $p_{th}$. For this, it is essential that the decoder has low time complexity; if the clock-rate of the quantum computer becomes limited by the decoder, the advantages of increasing the system size could be compromised.

Arguably, the most popular decoder for surface codes is the Minimum-Weight Perfect Matching (MWPM) decoder \cite{dennis2002topological}. The basic principle behind this approach is to identify the \emph{lowest weight} error configuration that can produce the syndrome. In general this is a good approximation to the optimal maximum likelihood decoder \cite{bravyi2014efficient}. For a toric code that only suffers random Pauli noise, the optimal code threshold is $p_{th} = 10.9\%$, whereas the MWPM decoder has $p_{th} = 10.3\%$. The minimum-weight matchings are found by constructing a fully connected graph between nodes of the syndrome, which leads to a cubic worst-case time complexity of $\mathcal{O}(n^3)$, where $n$ is the number of qubits in the system \cite{kolmogorov2009blossom}. Fowler has proved that the matching problem can be solved in average $\mathcal{O}(1)$ time, but only at sufficiently low error rates, and the worst-case complexity remains significant \cite{fowler2013minimum}. 

Many other decoding algorithms have been developed \cites{duclos2013fault, hutter2015improved, watson2015fast, tuckett2018ultrahigh, kubica2019cellular, torlai2017neural, varsamopoulos2017decoding}. Here, we build on top of a recently proposed decoder called the Union-Find (UF) decoder. It combines a very low time complexity with a high threshold \cite{delfosse2017linear, delfosse2017almost} making it a practical solution for real devices. 
For a toric code with only random Pauli noise, The Union-Find decoder has a threshold of $p_{th} = 9.9\%$ and worst-cast time complexity $\mathcal{O}(n\alpha(n))$, where $\alpha$ is the inverse of Ackermann's function. For any physical feasible amount of qubits, this value is $\alpha(n) \leq 3$, leading to an ``almost-linear'' time complexity.

We propose here a modification of the UF decoder that improves the heuristic for minimum-weight matching. The modified decoder, which we dub the \emph{Union-Find Node-Suspension decoder} (UFNS), achieves near MWPM thresholds while retaining a quasilinear time complexity. In section \ref{sec:surfacecode} we introduce the surface code. In section \ref{sec:ufbb} we describe the modified algorithm and its motivation. We discuss the complexity of the algorithm in section \ref{sec:complexity} and compare its performance with other decoders in section \ref{sec:performance}.  
\section{The Surface Code}\label{sec:surfacecode}

The Union-Find Node-Suspension decoder proposed here has the same compatibility as its parent decoder, and is applicable to any surface code of any genus, with or without boundary, and to color codes. For simplicity, we only describe the standard implementation of the surface code without boundary.

\subsection{The toric code}

The \emph{toric code}, a topological code introduced by Kitaev \cite{kitaev2003fault}, is defined by arranging qubits on the edges of a square lattice with periodic boundary conditions. The code is denoted by $V,E,F$, respectively the set of vertices, set of edges, and the set of faces on the lattice. The toric code is defined to be the ground state of the Hamiltonian 
\begin{equation}
    H = -\sum_{v \in V} X_v -\sum_{f \in F} Z_f, 
\end{equation}
where operator $X_v$ is the product of Pauli $X$ operators on the qubits located on edges forming the vertex $v$, \emph{i.e.,} $X_v = \prod_{e \in v} X_e$, and $Z_f = \prod_{e \in f} Z_e$ is the product of Pauli $Z$ operators on the qubits located on edges of face $f$. The code space is spanned by the simultaneous "+1" eigenstate of all operators $X_v$ and $Z_f$. These operators, together with any possible product of them, are the \emph{stabilizers} of the code, and form the stabilizer group $S$. This topology encodes the logical operators in the torus' non-trivial cycles. Errors, below a certain threshold, will only introduce local effects and do not change these cycles.

\subsection{Error model}
For simplicity, we will only consider i.i.d. phase-flip errors, where each qubit is subjected to a $Z$ error with probability $p_Z$. Due to \emph{lattice duality}, where the vertices and faces of the lattices can be interchanged, the error detection and correction of bit-flip errors is identical. 
Additionally, any qubit may be \emph{erased} from the system with probability $p_e$. The set of erased qubits is denoted with $\varepsilon$. This \emph{erasure} is detectable, such that we can replace or reinitiate all erased qubits, which corresponds to a random Pauli error after measurement. 

\subsection{Error correction}
Error correction is proceeded by measuring a set of independent stabilizers of the code, \emph{i.e.,} the operators $X_v$ and $Z_f$. For a set of phase-flip errors $E_Z = \{I,Z\}^{\otimes n}$, the stabilizers $X_v$ that anticommute with the error return a non-trivial outcome. The set of non-trivial eigenvalues of the stabilizers is called the syndrome $\sigma$ of the code. Given the measured $\sigma$, and optionally the known erasure $\varepsilon$, it is the task of the decoder to find the correction operator $\mathcal{C}(\sigma, \varepsilon)$. When the correction operator is applied, the code is returned to the code space, \emph{i.e.} $\mathcal{C}(\sigma, \varepsilon)E_Z \in S$. The error is corrected up to a stabilizer. The mapping of measured syndrome to the correction is thus not one-to-one, and it is up to the decoder to choose the most appropriate correction. 
\section{Union-Find decoder}\label{sec:unionfind}

The Union-Find decoder \cite{delfosse2017linear, delfosse2017almost} maps each syndrome $\sigma_i$ to a so-called non-trivial vertex $v_i$ in a non-connected graph on the code lattice, and grows clusters $c$ that form a connected graph $c(\m{V}, \m{E})$ of vertices $\m{V}$ and edges $\m{E}$ locally, by iteratively adding a layer of edges and trivial vertices to existing clusters, until all clusters have an even number of non-trivial syndrome vertices. Then, a spanning tree is built for each cluster, and every tree is traversed until all non-trivial syndrome vertices are paired and linked by a path, which is the correcting operator $\mathcal{C}$. By growing the clusters of vertices in order of their sizes, which is dubbed the \emph{weighted growth} version of the Union-Find decoder, the threshold is reported to increase from $9.2\%$ to $9.9\%$ compared to the non-weighted variant for the toric code. 

The complexity of the Union-Find decoder is driven by the merging between clusters. For this the algorithm uses the Union-Find or disjoint-set data structure \cite{tarjan1975efficiency}. The function \codefunc{Find} is used to traverse the vertex-tree of the cluster to the distinctive root element to find the parent cluster of a given vertex. If a newly added vertex has a different root, the vertex trees are merged by \codefunc{Union}. 
\section{Union-Find Node-Suspension decoder}\label{sec:ufbb}

In this section, we describe the \emph{Union-Find Node-Suspension} decoder, which increases the Union-Find decoder's performance by improving its heuristic for minimum-weight matching. We first introduce the concept of the potential matching weight in \Cref{sec:matchingweight}. We describe the data structure required for this decoder in \Cref{sec:nodeset}, and the necessary calculations performed on this data structure in \Cref{sec:paritydelaysus,sec:nodejoin,sec:inversion}. The pseudocode is included in \Cref{sec:pseudocode}. 

\Figure[htb](topskip=0pt, botskip=0pt, midskip=0pt){tikzfigs/tikz-figure0.pdf}{
    A cluster with vertices $\{v_0, v_1, v_2\}$ with potential matching weights $\{2, 3, 2\}$. The line style and color of the colored edges correspond to the matching in the hypothetical union with an external vertex $v'$ of the same line style and color.\label{fig0}}

\subsection{Potential matching weight}\label{sec:matchingweight}
%In the following we give some intuition into the improvement of the Union-Find Balanced Bloom decoder upon the original Union-Find decoder. 
% We compared the ratio of the matchings between the MWPM decoder and our own implementation of the UF decoder, averaged over many simulations, and found that UF matching weight has a constant prefactor of $\sim 1.043$ over the minimum weight for the toric code (\Cref{comp_weight}). From this, we suspected that a decreased matching weight is a heuristic for an increased threshold. Within the context of the UF decoder, the matching weight may be decreased by prioritizing the growth of vertices with low PWM's within the cluster. 

Consider the cluster with index $i$ containing the set of non-trivial vertices $V_i=\{v_0,v_1,v_2\}$ and set of edges $E_i=\{(v_0,v_1), (v_1, v_2)\}$ of \Cref{fig0}. Now let us investigate the weight of a matching if an additional non-trivial vertex $v'$ is connected to the cluster. If $v'$ is connected to $v_0$ or to $v_2$, then the resulting matching has a total weight of 2: $(v',v_0)$ and $(v_1,v_2)$, or $(v_0,v_1)$ and $(v_2,v')$. However, if $v'$ is connected to vertex $v_2$, then the total weight is 3: $(v', v_1)$ and $(v_0, v_2)$. Inspired by this idea, we introduce the concept of potential matching weight (PMW) of a vertex. 

\begin{definition}\label{def:pmw}
    % For, the hypothetical merger with another odd-parity cluster $V_j, E_j$ on the edge $(v_i, v_j)$, with $v_i\in V_i$ and  $v_j \in V_j$, outputs an even-parity cluster with edges $E_{ij} = E_i \cup E_j \cup (v_i, v_j)$ in which there exists a matching $\m{C}_{(v,v')} \subseteq E_{ij}$ between syndromes internal to the cluster.
    % Let the Potential Matching Weight (PMW) of vertex $v_\alpha \in V_\alpha$ in an odd-parity cluster $\alpha$ with vertices $V_\alpha$ and edges $E_\alpha$ be
    % \begin{equation}
    %   PMW(v_\alpha) = \abs{\m{C}_{(v_\alpha,v_\beta)} \cap E_\alpha} + 1,
    % \end{equation}
    % where matching $\m{C}_{(v,v')} \subseteq E_{ij}$ is between the syndrome vertices internal to the even-parity cluster with edges $E_{ij} = E_\alpha \cup E_\beta \cup (v_\alpha, v_\beta)$, after a hypothetical merger of cluster $\alpha$ with another odd-parity cluster $V_\beta, E_\beta$ on the edge $(v_\alpha, v_\beta)$, with $v_\alpha\in V_\alpha$ and  $v_\beta \in V_\beta$
    Let there be a hypothetical merger between odd cluster $\alpha$ of vertices $V_\alpha$ and edges $E_\alpha$, and odd cluster $\beta$ of $V_\beta$ and $E_\beta$, on the edge $(v_\alpha, v_\beta)$, where $v_\alpha \in V_\alpha$ and $v_\beta \in V_\beta$. In the merged even cluster with edges $E_{\gamma} = E_\alpha \cup E_\beta \cup (v_\alpha, v_\beta)$, there is a matching $\m{C}_{(v,v')} \subseteq E_{\gamma}$  between the syndrome vertices internal to the cluster. The \textbf{Potential Matching Weight} (PMW) of vertex $v_\alpha$ is then defined as
    \begin{equation}
      PMW(v_\alpha) = \abs{\m{C}_{(v_\alpha,v_\beta)} \cap E_\alpha} + 1.
    \end{equation}
\end{definition}

In other words, the PMW is a vertex-specific predictive heuristic to the matching weight, assuming a union occurs in the next growth iteration. The PMW can be utilized by prioritizing the growth of vertices with low PMW such that there is an increased probability of mergers between clusters on edges connected to these vertices, and there is an increased probability in a lower matching weight. However, the PMWs' calculation within a cluster is not a trivial task, especially for clusters of increasingly larger size, as all edges of a cluster must be considered in its calculation. Furthermore, the PMWs within a cluster change due to cluster growth and mergers, both of which occur more frequently as the system size increases. For this reason, the scaling of the PMW computation is vital to the decoder. 

\Figure[htb](topskip=0pt, botskip=0pt, midskip=0pt){tikzfigs/tikz-figure1.pdf}{
    The cluster of \Cref{fig0} after two rounds of prioritized growth of $v_0$ and $v_2$. There are regions of vertices that are either interior elements or have equal potential matching weights, represented as nodes with different node radii in the node-tree $\nset$. \label{fig1}}

\subsection{Node-Suspension data structure}\label{sec:nodeset}

Fortunately, the PMW calculation is quite efficient by the introduction of a new data structure. Consider the cluster of non-trivial vertices $V_i=\{v_0,v_1,v_2\}$ and edges $E_i = \{(v_0,v_1), (v_1, v_2)\}$ from \Cref{fig0}. We had found previously that vertices $v_0, v_2$ have a lower PMW compared to $v_1$ by 1 edge. The growth of $v_0$ and $v_2$ are thus prioritized, such that new vertices are added to the cluster on the boundary of $v_0$ and $v_2$. If all newly added vertices are trivial, the cluster is now as in \Cref{fig1}. If we repeat the PMW calculation, we now find that the PMWs in the new vertices connected to $v_0$ are equal. The same is true for vertices connected to $v_2$. 
\begin{definition}\label{def:vertextree}
    Let the vertex-tree $\vset_i$ be a connected acyclic subgraph of the graph of a cluster $G(V_i, E_i)$.   The vertex-tree $\vset_i$ includes all vertices $V_i$ and a minimum number of edges in $E^\vset_i \subseteq E_i$. 
\end{definition}
\begin{definition}
  Let the node-tree $\nset_i$ be a partition of the vertex-tree $\vset_i$, such that each element of the partition --- a \textbf{node} $n$ --- consists of a set of adjacent vertices that lie within a certain distance --- the \textbf{node radius} $r$ --- from the \textbf{primer vertex}, which initializes the node and lies at its center. The node-tree is a directed acyclic graph, and its edges $\m{E}_i$ have lengths equal to the distance between the primer vertices of neighboring nodes. 
\end{definition}

The concept of primer vertices is easily understood when considering non-trivial vertices of the syndrome $\sigma$. Suppose every non-trivial vertex is the primer of a node, the weight of a matching in $\vset_i$ equal to the weight of the same matching in $\nset_i$. Furthermore, for every node of the node-tree, all vertices that lie at distance $r$ to the primer vertex are either boundary vertices to the cluster and have equal PMW, or lie within the radius of another node. For the example in \Cref{fig1}, the PMW of all boundary vertices of $n_0$, for simplicity just the PMW of $n_0$, is $\floor{r_0} + (n_1, n_2) + 1$. The partition from $\vset$ to $\nset$ thus allows us to compute the PMW on a reduced tree. 

\Figure[htb](topskip=0pt, botskip=0pt, midskip=0pt){tikzfigs/tikz-figure2.pdf}{
    Two different types of nodes. Syndrome-nodes $s$ have a non-trivial vertex or syndrome at its center. Vertices that lie on the radii of two existing nodes initialize a junction-node $j$ in the node-tree.\label{fig2}}

\Figure[hbt](topskip=0pt, botskip=0pt, midskip=0pt){tikzfigs/tikz-figure3.pdf}{
    The relevant data structures. \emph{(a)} The cluster-tree of the Union-Find data structure. The path from a vertex to the root of the cluster-tree is traversed to find the root element in order to differentiate between clusters. The root node of the node-tree is now additionally stored at the root of the cluster-tree. \emph{(b)} The vertex-tree $\vset$ with 9 non-trivial vertices. As $\vset$ is strictly acyclic, the cluster's edges must be maintained such that no cycles are created. This is done during growth by removing edges (red dotted lines) if a cycle is detected. \emph{(c)} The node-tree $\nset$, which currently has the same number of elements as $\vset$, as all vertices are non-trivial. Two depth-first searches are required to compute node parities (head recursively) and delays (tail recursively) in $\nset$.\label{fig3}}

All non-trivial vertices serve as primers for nodes that are called \textbf{syndrome-nodes} $s$. However, not all primer vertices are non-trivial vertices of the syndrome. If two non-trivial vertices are located an even Manhattan distance on the lattice, the growth of their clusters can simultaneously reach some vertex that lies on equal radii of the associated nodes, such as in \Cref{fig2}. For this reason, such vertices serve as primers of a different type of node --- a \textbf{junction-node} $j$ --- in the merged node-tree. 

The calculation of the PMW on the node-tree $\nset$ rather than the vertex-tree $\vset$ offers a reduction in the cost. However, it is still no trivial task as the entire tree must be considered for the calculation in every node. Instead, we will compute for the \textbf{node suspension} $n_s$ --- the number of growth iterations needed for a node to reach the maximum PMW in the node-tree --- which relates closely to the PMW. For example, the node suspension for the nodes $\{n_0, n_1, n_2\}$ associated with the vertices $\{v_0, v_1, v_2\}$ in \Cref{fig0} is $\{0, 2, 0\}$, and $\{0, 1, 0\}$ in \Cref{fig1}.

The Node-Suspension data structure does not replace but coexists with the Union-Find data structure. Additional to the the Union-Find data structure's cluster-trees of distinct roots, we store for every cluster the node-tree $\nset_i$ by its root node. For this, we need to maintain the reduced set of edges $E^\vset_i \subseteq E_i$ of the vertex-trees $\vset_i$ for every cluster, which can be done in constant time (see Algorithm \ref{algo:ufbb}). In the UF decoder, vertex-trees $\m{V}_i$ are not maintained, such that the graph associated with each cluster is not acyclic \cite{delfosse2017almost}. Instead, a spanning forest $F$ of all clusters is created \cite{delfosse2017linear} after growth. Each connected element within $F$ is also an acyclic graph. The difference is that while a single depth-first search or breath-first-search creates $F$, $\vset$ is equivalent to multiple breadth-first searches from each non-trivial vertex within the cluster, where the search of every breadth occurs during a growth iteration. The relevant data structures are depicted in \Cref{fig3}. 


\subsection{Node parity, delay, and suspension}\label{sec:paritydelaysus}

The Node-Suspension data structure allows for calculating the node suspension of all nodes in a node-tree $\nset$ by two intermediate steps. In each step, a depth-first-search (DFS) of $\nset$ is applied from its root node $r$ (\Cref{fig3}c).

In the first DFS, we calculate for the \textbf{node parity} $n_p$ --- the number of descendant syndrome-nodes of a node modulo 2 --- via a tail-recursive function, which is only dependent on the node parities of the children nodes of a node. The node parity is defined differently per node type:
\begin{align}\label{eq:nodeparity}
    s_p &= \hspace{.6cm}\big( \sum_{\mathclap{n \in \text{ children of } s}} (1-n_p) \big ) \bmod 2,\\
    j_p &= 1 - \big(\sum_{\mathclap{n \in \text{ children of } j}} (1-n_p) \big) \bmod 2.
\end{align}

In the second DFS, we calculate for the difference in node suspension of a node $n$ with its parent $m$; $\delta = n_s - m_s$. We can choose an arbitrary \textbf{node delay} $n_d$ --- the node suspension minus the maximum node suspension in the node-tree --- for the root node $r$ such as $r_d=0$ and add the suspension difference $\delta$ during each step to obtain $n_d$ for every node. This node delay of a node $n$ is only dependent on the node radii of itself and its parent $m$, the length of edge $(n,m)$, and its parity $n_p$. 
\begin{multline}\label{eq:delayequation}
    n_d = m_d + \bigg \lceil 2C\big(\ceil{n_r} - \floor{m_r + n_r \bmod 1}\\
    - (-1)^{n_p}\abs{(n,m)}\big) - 2(n_r - m_r) \bmod2 \bigg \rceil
\end{multline}
Here, the \textbf{inversion constant} $C$ deals with the inversion of node parities in a node-tree during merges of clusters explained in \ref{sec:nodejoin}. The node suspension is then related to the node delay by
\begin{equation*}
    n_s = n_d - \max_{x \in \nset}{x_d}. 
\end{equation*}
The maximum node delay can be maintained during the second DFS of the node-tree, and the node suspension itself is calculated during cluster growth. A single growth iteration, which is applied in the UF decoder by adding half-edges to all boundary vertices of the cluster, is now replaced by another DFS of $\nset$. During this DFS, we calculate the suspension $n_s$ for a node, and conditionally grow it - adding half-edges to the boundary vertices in the current node and adding 1 to its radius $n_r$ --- if $n_s = 0$. This requires us to save the list of boundary vertices to each node (\Cref{fig3}c). When all $n_s$ in $\nset$ are zero, all nodes are grown simultaneously within the same iteration. 

If the node-tree does not change after a growth iteration, which is the case if no mergers occur between clusters, the node suspensions decrease in an expected manner: For all nodes that are not suspended from growth, their node suspensions decrease with 1 in the next growth iteration. Due to this behavior, we can reuse the node delays $n_d$ to calculate $n_s$ for the next growth iteration by introducing another node parameter $n_w$, the number of iterations a node has \textbf{waited}. Each time a node is suspended from growth, we add 1 to $n_w$. The node suspension in subsequent iterations is then
\begin{equation}\label{eq:suspension}
    n_s = n_d - \max_{x \in \nset}{x_d} - n_w. 
\end{equation}
Note that we have not stated which node in $\nset$ should be the root node. In fact, any node in $\nset$ could have been picked as the root of the node-tree. As long as the DFS of cluster growth is performed in the same direction as the DFSs of the parity and delay calculations. If no cluster mergers occur, the node delays can be reused in the node suspension calculation prior to node growth. 
The node-tree is constructed by storing all neighbors of a node to a list. This way, the DFSs' direction can be determined by simply saving the root node, the starting point of the DFSs, to the cluster. All node variables are depicted in \Cref{fig3}c. 
%In the next section, we expand upon this idea of "reusing" some intermediate parameters to calculate the node suspensions after a cluster merger.  


\subsection{Joining node-trees}\label{sec:nodejoin}

In the Union-Find (UF) algorithm, odd parity clusters of an odd number of non-trivial vertices, --- elements of $\sigma$ --- grow in size repeatedly and merge with other clusters until all clusters are even. During these mergers, the node-trees of the Node-Suspension data structure must also be combined. Let us now first make a clear distinction between the merging protocols of the underlying data structures; the clusters-trees of the UF data structure are merged with the \codefunc{Union} function, whereas the node-trees are merged with a separate \codefunc{Join} function. After a join of multiple node-trees, the node suspensions within the combined node-tree change. Therefore, \codefunc{Join} protocol's focus is to minimize the DFSs of the recalculation of the node parity and delays in the combined node-tree. 

First, note that as a cluster of even parity has an even number of non-trivial vertices, its node-tree has an even number of syndrome-nodes. For these even node-trees, the concept of PMW does not exist, as the matching can be made within the node-tree. Consequently, node suspension, parity, and delays are undefined when two odd node-trees join to an even node-tree. 
%Thus, if two odd clusters merge into an even cluster, we don't know and do not care about its node suspensions. 

\Figure[hbt](topskip=0pt, botskip=0pt, midskip=0pt){tikzfigs/tikz-figure4.pdf}{
    \emph{(a)} An odd cluster $\nset_o=\{n_1, n_2, n_o\}$ with root $n_1$ joins with an even cluster $\nset_e=\{n_3, n_e\}$ with root $n_3$ on nodes $n_o, n_e$, respectively, to a joined node-tree. If we choose to \emph{(b)}, make $n_e$ a child of $n_o$, the parities and delays the sub-tree of $\nset_o$ can unchanged, and we only have to perform partial parity and delay calculations over the sub-tree of $\nset_e$. If we choose to \emph{(c)}, make $n_o$ a child of $n_e$, parities and delays have to be recalculated in the entire joined node-tree. \label{fig4}}

The second type of merger is between an even and an odd cluster. The combined cluster is odd, and its growth is continued. Thus its node suspensions must be computed. Consider the example of odd node-tree $\nset_o$ and even node-tree $\nset_e$ that are to be joined on nodes $n_o\in \nset_o$ and $n_e \in \nset_e$ (\Cref{fig4}\emph{a}). If $\nset_o$'s root is kept as the root of the joined node-tree (\Cref{fig4}\emph{b}), $n_e$ is to be a child node of $n_o$. As $\nset_e$ contains an even number of syndrome-nodes, the node parities in $\nset_o$ do not change. Hence, the node parity DFS is only necessary in the sub-tree $\nset_e$, which now has $n_e$ as sub-root. Furthermore, as the node delay is only dependent on its own properties and its parent's, the node delay DFS is also only required from node $n_e$ and within the sub-tree of $\nset_e$. These so-called \textbf{partial} DFSs of the node-tree are precisely what was required, as the node parity and delays in $\nset_e$ were undefined. Alternatively, if $\nset_e$'s root becomes the root of the combined tree (\Cref{fig4}\emph{c}), an odd number of syndrome-nodes are attached to $n_e$, such that the parities of nodes on the path from $n_e$ to the root are changed. Such a join would require the DFSs on the entire combined node-tree to calculate for node parities and delays. Thus, a simple rule is always to keep the root of the odd node-tree, which we dub \textbf{Odd-Rooted Join}.

In addition, a cluster can be subjected to multiple mergers within the same growth iteration, during which the parity of the merged cluster changes dependent on the number of mergers and the parities of the clusters involved. The DFSs related to the parity and delay calculations must, for this reason, not be initiated directly after the joining of node-trees. After all, it may be possible for the cluster to merge again such that the parities and delays become invalid. To prevent these redundant calculations, sub-roots of the even sub-trees are stored to a list $\m{S}$ at the root of the node-tree (\Cref{fig3}\emph{c}). When multiple mergers occur, the root node that stores the now redundant sub-roots is replaced by a new root with new $\m{S}$. If a cluster is selected for growth, we check for the sub-roots in $\m{S}$ at the new root node and initiate the DFSs from these sub-roots. We call this the \textbf{Root List $\m{S}$ Replacement}. 

\Figure[htb](topskip=0pt, botskip=0pt, midskip=0pt){tikzfigs/tikz-figure5.pdf}{
    The node suspension values for nodes for 3 odd node-trees $\{\nset_1, \nset_2, \nset_3\}$ of 3 nodes that grow and join into a single node-tree. \emph{(a)} Node suspensions are calculated by setting $C=1$ in equation \eqref{eq:delayequation}. In step 1, the growth in each of the three node-trees' outer nodes is prioritized, and the node-trees merge. In step 2, the recalculation of the joined node-tree is performed. Parities within the sub-tree of $\nset_2$ are now inverted, and the suspension in these nodes have doubled. \emph{(b)} Node suspensions are calculated by setting $C=\nicefrac{1}{2}$. Now the increase in node suspensions after parity inversion is halved.\label{fig5}}

\subsection{Parity inversion}\label{sec:inversion}
An unfortunate effect of the Node-Suspension data structure, which we dub \textbf{Parity Inversion}, causes a decrease in the algorithm's performance as the lattice size is increased. We will demonstrate this effect through the example in \Cref{fig5}\emph{a}. Consider three instances of the node-tree of \Cref{fig0}; $\nset_a, \nset_b, \nset_c$, positioned near each other on the lattice. For each node-tree, if the middle node is suspended from growth for two iterations, all nodes have the same Potential Matching Weight. However, in the current example, the node-trees $\nset_a, \nset_b, \nset_c$ merge after 1 iteration. The combined node-tree is odd. Thus, we recalculate the node parities and delays to find that the parities in the partition of the node-tree containing the nodes of $\nset_b$ have been inverted, and the node suspensions in this partition have doubled from before node suspensions before the merger. If the next merging event occurs on the node with the doubled node suspension, the matching weight may be larger compared to the original UF decoder, which defies the goal of Node-Suspension to decrease the weight.

This defines a trade-off in the Node-Suspension data structure; a node must wait as many iterations as it is suspended to reach equilibrium in Potential Matching Weight in the node-tree, but after Parity Inversion, the node suspension for previously prioritized nodes increases linearly with the number of iterations waited by the suspended nodes pre-inversion. As a compromise, we redefine the node suspension as \textbf{half} the number of growth iterations needed for all nodes in the node-tree to reach equal PMW. This can be done by setting $C=0.5$ in Equation \eqref{eq:delayequation}. Nevertheless, as more inversions occur, the maximum node suspension in the node-tree increases, and it becomes more and more unlikely for a cluster to actually reach zero node suspension in all nodes. The number of inversions is directly related to the number of merging events, and thus the size of the lattice. The performance to improve the heuristic for minimum weight matching thus decreases for larger lattices. 


\subsection{Potential matching weight}\label{sec:matchingweight}

In the following we give some intuition into the improvement of the Union-Find Balanced Bloom decoder upon the original Union-Find decoder. Consider the cluster containing the set of edges $\m{E}$ and set of non-trivial vertices $\mathcal{V}=\{v_0,v_1,v_2\}$ in Figure \ref{fig0}. Now let us investigate the weights of a matching if an additional non-trivial vertex $v'$ is connected to the cluster. If $v'$ is connected to $v_0$ or to $v_2$, then the resulting matching has a total weight of 2: $(v',v_0)$ and $(v_1,v_2)$, or $(v_0,v_1)$ and $(v_2,v')$. However, if $v'$ is connected to vertex $v_2$, then the total weight is 3: $(v', v_1)$ and $(v_0, v_2)$. Inspired by this idea, we introduce the concept of potential matching weight (PMW) of a vertex. 

\Figure[htb](topskip=0pt, botskip=0pt, midskip=0pt){tikzfigs/tikz-figure0.pdf}{Unbalanced matching weight in cluster vertex set $\mathcal{V}$. The matching edges (dashed) correspond to the position of $v'$.\label{fig0}}

\begin{definition}\label{def:pmw}
    Consider an odd-parity cluster $c_i$ containing a vertex $v$. The Potential Matching Weight (PMW) of the vertex $v$ is the matching weight in the subset of edges of an odd-parity cluster $c_i$ in a hypothetical union with another cluster $c_j$ in the next growth iteration, where the merging boundary edge is supported by $v$. 
    \begin{equation}
      PMW(v) = \abs{\m{C} \cap \m{E}} + 1 |\text{ merge to even cluster on } v
    \end{equation}
\end{definition}
In other words, the potential matching weight is a vertex-specific predictive heuristic to the matching weight assuming a union in the next growth iteration. However, the calculation of the potential matching weight is seemingly not as straight forward, especially for clusters of increasingly larger size. Furthermore, if the potential matching weight is to be calculated for every vertex with boundary edges in all clusters in every growth iteration, the algorithm's time complexity would increase dramatically. Luckily, these calculations can be reduced to be performed on a set of \emph{nodes} in each cluster. 


\subsection{Priority-Node data structure}\label{sec:nodeset}

Fortunately, the scale of the PMW calculation and number of recalculations can be minimized by the introduction of a new data structure. 
Consider the previous example of Figure \ref{fig0}. The growth of $v_0$ and $v_2$ are prioritized for a single iteration, such that new (trivial) vertices are added to the cluster on the boundary of $v_0$ and $v_2$. If we repeat the process of PMW calculation, we now find that the PMW's in the new vertices connected to $v_0$ are equal, and the same is true for $v_2$. These regions of equal PMW vertices, together with the vertices they have been added to which are now interior elements of the associated graph, can be used as nodes in a simplified tree representation of the cluster (see Figure \ref{fig1}). 

\begin{definition}
  The node-tree of a cluster $\nset$ is a partition of the cluster such that each element of the partition - a node $n$ - consists of a set of adjacent vertices that are either interior elements in the associated graph or have equal PMW. 
\end{definition}

\Figure[htb](topskip=0pt, botskip=0pt, midskip=0pt){tikzfigs/tikz-figure1.pdf}{The cluster of Figure \ref{fig0} after a round of prioritized growth of $v_0$ and $v_2$. There are regions of vertices that are either interior elements or have equal potential matching weights, which can be represented as nodes with different node radii in the node-tree $\nset$. \label{fig1}}

The boundary vertices of equal PMW of a node which has grown prioritized are now located at a larger distance from the interior non-trivial vertex or syndrome than those of delayed growth. This property of the node can be stored as the \textbf{node radius} $nr$, which can be used to calculate the PMW of a node without $\vset$. For the example in Figure \ref{fig1}, the PMW of $n_0$ is $nr_0 + (n_1, n_2) + 1$.

The calculation of the PMW on the node-tree $\nset$ rather than the vertex-tree $\vset$ offers a reduction in the cost. However, it is still no trivial task as the entire tree must be considered for the calculation in each node. Instead, we will compute for the \textbf{node suspension}, which relates closely to the PWM. 
\begin{definition}
  The node suspension is the number of growth iterations needed to reach the maximum PMW in the node-tree. 
\end{definition}
For example, the node suspension for $\nset = \{n_0, n_1, n_2\}$ of the cluster in Figure \ref{fig0} is $\{0, 2, 0\}$ and $\{0, 1, 0\}$ in Figure \ref{fig1}, as half-edges are added per growth iteration. 

There are different types of nodes which determines the way in which the node suspension is calculated, which is detailed in Section \ref{sec:paritydelaysus}. The nodes with a non-trivial vertex or syndrome at its center is dubbed a \textbf{syndrome-node} $sn$. These are the nodes we have encounted so far in Figures \ref{fig0} and \ref{fig1}. If two syndromes are located an even manhattan-distance on the lattice, their growths will simultaneously reach some vertex that lies on equal radii of both associated nodes, such as in Figure \ref{fig2}. For this reason, such vertices will always initialize a second type \textbf{junction-node} $jn$ in the combined node-tree.

\Figure[htb](topskip=0pt, botskip=0pt, midskip=0pt){tikzfigs/tikz-figure2.pdf}{Two different types of nodes. Syndrome-nodes $sn$ have a non-trivial vertex or syndrome at its center. Vertices that lie on the radii of two existing nodes initialize a junction-node $jn$ in the node-tree.\label{fig2}}

The Priority-Node data structure, which exists to prioritize zero node suspension growth, does not replace but coexists with the Union-Find data structure, which allows for differentiation between clusters. However, the Priority-Node data structure does require change in the Union-Find data structure. The node-trees are connected, acyclic graphs, which deponds upon that the vertex-trees of the Union-Find data structure are also always acyclic. The maintenance of these acyclic vertex-trees can be done in constant time and is equivalent to the post-growth depth-first-search (DFS) of the cluster as in the original UF decoder. 

\subsection{Node parity, delay and suspension}\label{sec:paritydelaysus}

The Node-Suspension data structure allows for the calculation of the node suspension of all nodes in a node-tree $\nset$ by two intermediate steps. In each step a depth-first-search (DFS) of $\nset$ is applied, such that the calculation can be performed in linear time to the node-tree dimension.

\Figure[b](topskip=0pt, botskip=0pt, midskip=0pt){tikzfigs/tikz-figure3.pdf}{Two depth-first searches on $\mathcal{N}$ to compute node parities (head recursively) and delays (tail recursively).\label{fig3}}

In the first DFS, we calculate for the \textbf{node parity} $np$ --- the number of descendant syndrome-nodes of a node modulo 2 --- via a tail recursive function, which is only dependent on the node parities of the children nodes of a node.
\begin{align}\label{eq:nodeparity}
    snp &= &\big( \sum_{\mathclap{n_\gamma \in \text{ children of } sn}} (1-np_\gamma) \big ) \bmod 2 \hspace{1em}  \\
    jnp &= 1 - &\big(\sum_{\mathclap{n_\gamma \in \text{ children of } jn}} (1-np_\gamma) \big) \bmod 2 \hspace{1em}
\end{align}

In the second DFS, we calculate for the difference in node suspension of a node $n_\beta$ with its parent $n_\alpha$; $\delta = ns_\beta - ns_\alpha$. We can choose an arbitrary \textbf{node delay} $nd$ --- the node suspension minus the maximum node suspension in the node-tree --- for the root node such as $nd_r=0$ and add the suspension difference $\delta$ during each step to obtain $nd$ for every node. This node delay of a node $n_\beta$ is only dependent on the node radii of itself and of its parent $n_\alpha$, the length of edge $(n_\beta, n_\alpha)$ and its parity $np_\beta$. 
\begin{multline}\label{eq:delayequation}
    nd_\beta = nd_\alpha + \Bigg \lceil 2k_{inv}\bigg(\ceil{nr_\beta} - \floor{nr_\alpha + nr_\beta \bmod 1}\\
    - (-1)^{np_\beta}\abs{(n_\beta,n_\alpha)}\bigg) - 2(nr_\beta - nr_\alpha) \bmod2 \Bigg \rceil
\end{multline}
Here, the \textbf{inversion factor} $k_{inv}$ is a constant that deals with the inversion of node parities in a node-tree during merges of clusters, explained in \ref{sec:nodejoin}. 

There is a final step in calculating the node suspension from the node delay, which are related by
\begin{equation}\label{eq:suspension}
    ns = nd - \max_{nd_i \in \nset} nd_i - nw, 
\end{equation}
where $nw$ is equal to the number of iterations a node has \textbf{waited} or has been suspended from growth. The maximum node delay can be maintained during the second DFS of the node-tree, and the node suspension itself is calculated during the DFS related to the growth of the cluster, such that this step is not counted towards the node suspension calculation. 

A single growth iteration, which is applied in the original UF decoder by adding half-edges to all boundary vertices of the cluster, is now replaced by another DFS of $\nset$. During this DFS, a node is conditionally grown -- adding half-edges to the boundary vertices in the current node and adding 1 to its radius $nr$ -- if $ns$ is equal to zero. If not, suspend the node growth, add 1 to $nw$ and continue with the DFS. A subsequent growth iteration now does not require the two DFS's related to the calculation of $np$ and $nd$, provided that no union between clusters has occurred. Hence, the Node-Suspension data structure enables us to calculate the node suspension across multiple growth iterations based on a single calculation of the node parity and delay. When all $ns$ in $\nset$ are zero, all nodes are bloomed simultanenously within the same iteration. 

Note that we hadn't stated which node in $\nset$ should be the root node. In fact, any node in $\nset$ could have been picked as the root of the node-tree. This property, together with the constancy of $np$ and $nd$ in between cluster unions, allows us to define a set of rules for the merging of node-trees.  
\subsection{Joining node-trees}\label{sec:nodejoin}

\Figure[htb](topskip=0pt, botskip=0pt, midskip=0pt){tikzfigs/tikz-figure4.pdf}{(a) An odd cluster $\nset^o=\{s_1, s_2, s_3\}$ with root $n^o_r = s_1$ joins with an even cluster $\nset^e=\{s_4, s_5\}$ with root $n^e_r=s_4$ on nodes $s_3, s_5$, respectively, to a new set $\nset$ with subsets $'\nset^e$ and $'\nset^o$.  If we choose to (b), make $s_5$ a child of $s_3$, the parities and delays in $'\nset^o$ can be reused, and we only have perform partial parity and delay calculations over $'\nset^e$. If we choose to (c), make $s_3$ a child of $s_5$, parities and delays have to be recalculated over both $'\nset^2e$ and $'\nset^o$. \label{fig4}}

In the Union-Find algorithm, clusters of an odd number of non-trivial vertices or syndromes grow in size iteratively and merge with other clusters until all clusters are of even parity. During this process, the node-trees of the Node-Suspension data structure must also be merged. Let us now first make a clear distinction between the merging procotols of the underlying data structures; the vertex-trees of the UF data structure are merged with \codefunc{Union}, whereas the node-trees are merged with \codefunc{Join}. Once node-trees are joined, the node suspensions within the combined node-tree changes. The focus of the \codefunc{Join} protocol is therefore to minimize the DFS's of the recalculation of the node parity and delays in the combined node-tree. 

First, node that if a cluster of even parity has a even number of non-trivial vertices or syndromes, its node-tree has an even number of syndrome-nodes. For an even node-tree there are no PMW's as the matching can be made within the node-tree. Consequently, node suspension, parity and delays are undefined for an even node-tree. Thus, if two odd clusters merge into an even cluster, we don't know and do not care about its node suspensions. 

The second type is the merge between an even and an odd cluster. The combined cluster is odd and it will continue to grow in size, thus its node suspensions must be computed. Now, consider the example of odd node-tree $\nset_o$ and even node-tree $\nset_e$ that are to be joined on nodes $n_o\in \nset_o$ and $n_e \in \nset_e$ (Figure \ref{fig4}\emph{a}). If the root of $\nset_o$ is kept as the root of the joined node-tree (Figure \ref{fig4}\emph{b}), $n_e$ is to be a child node of $n_o$. As $\nset_e$ contains an even number of syndrome-nodes, the node parities in $\nset_e$ do not change. The DFS of the node parity recalculation is only necessary in $\nset_e$, which now has $n_e$ as sub-root. Furthermore, as the node delay is only dependant on the properties of a node and of its parent node, the DFS of the node delay recalculation is also only required from node $n_e$ and is performed within $\nset_e$. These paritial DFS's of the even sub-tree are exactly what was required as the node parity and delays in $\nset_e$ were undefined. If the root of $\nset_e$ takes the role of the root of the combined tree (Figure \ref{fig4}\emph{c}), an odd number of syndrome-nodes are attached to $n_e$, such that the parities of nodes on the path from $n_e$ to the root are changed. Such a join would require the DFS's for the entire combined node-tree of the recalculation of node parities and delays. A simple rules is thus to always keep the root of the odd node-tree. 

Finally, a cluster can be subjected to merges with multiple other clusters within the same growth iteration, duing which the merged cluster may switch parity multiple times. The DFS's related to the recalculation of the node parities and delay must for this reason not be initiated directly of the the joining of node-trees. Instead, a pointer to the sub-root of the even sub-tree in the last odd-even join is stored at the root of the node-tree. The recalculation is then initiated just before cluster growth to prevent multiple recalculations over the same partitions of the node-tree. 

% to even -> node suspension undefined


% \Figure[htb](topskip=0pt, botskip=0pt, midskip=0pt){tikzfigs/tikz-figure5.pdf}{bla\label{fig5}}

% Equilibruim
\subsection{Parity inversion}\label{sec:inversion}

An unfortunate effect of the Node-Suspension data structure, which we dub \textbf{parity inversion}, causes a decrease in the performance of the algorithm as the lattice size is increased (see Figure \ref{threshold_ufbb}). We will demonstrate this effect through the example in Figure \ref{fig5}\emph{a}. Consider three instances of the node-tree of Figure \ref{fig0}; $\nset_a, \nset_b, \nset_c$, positioned closely to each other on the lattice. Every node in all node-trees have radius $\nicefrac{1}{2}$, and the node suspension in $n_1$ in each node-tree is 2. This means that if $n_1$ is suspended for two growth iterations, such as in Figure \ref{fig1}, all nodes have the same PMW. However, in the current example, the node-trees $\nset_a, \nset_b, \nset_c$ merge after 1 iteration. The merged cluster is odd, thus we recalculate the node parities and delays to find that the parities in the partition of the node-tree containing the nodes of $\nset_b$ have been inverted, and the node suspensions in this partition have increased dramatically. If the next merging event occurs on the node with the increased node suspension, the matching weight may be larger compared to the original UF decoder, which defies the current goal. 

\Figure[t!](topskip=0pt, botskip=0pt, midskip=0pt){tikzfigs/tikz-figure5.pdf}{
    The node suspension values for nodes for 3 odd node-trees $\{\nset_1, \nset_2, \nset_3\}$ of 3 nodes that grow and join into a single node-tree. (a)
    Node suspensions are calculated by setting $k_{inv}=1$ in equation \eqref{eq:delayequation}. In step 1, the growth in the outer nodes of each of the three node-trees are prioritized and the node-trees merge. In step 2, recalculation of the joined node-tree isperformed. Parities within the sub-tree of $\nset_2$ are now inverted, and the suspension in these nodes have doubled. (b) Node suspensions are calculated by setting $k_{inv}=\nicefrac{1}{2}$. Now the increase in node suspensions after parity inversion is halved.\label{fig5}}

This defines a trade-off in the node suspension; a node must wait as many iterations as it is suspended to reach equal PMW in the node-tree, but after a parity-inversion the node suspension for previously prioritized nodes increase linearly with the number of iterations waited by the suspended nodes pre-inversion. As a compromise, we redefine the node suspension as \textbf{half} the number of growth iterations needed for all nodes in the node-tree to reach equal PMW. This can be done in Equation \eqref{eq:delayequation} by setting $k_{inv}=0.5$.

Nevertheless, as more parity inversions occur, the maximum node suspension in the node-tree increases, and it becomes more and more unlikely for a cluster to actually reach zero node suspension in all nodes. The number of parity inversions is directly related to the number of merging events, and thus the size of the lattice. The performance to improve the heuristic for minimum weight matchings thus decreases for larger lattices. 


\subsection{Pseudocode}\label{sec:pseudocode}
\begin{algorithm}[htb]
  \BlankLine
  \KwData{A graph $G=(V,E)$, and syndrome $\sigma \subseteq V$}
  \KwResult{Correction set $\m{C}$}
  \BlankLine
  Initialize cluster vertex-trees, node-trees and table \emph{Support}.\;\label{algo:B1a}
  Create the list $\m{L}$ of odd clusters.\;
  \While(){$\m{L}$ is not empty}{
    Initialize the fusion list $\m{F}$ as an empty list.\;\label{algo:B1b}
    \For(){cluster $\in\m{L}$ \label{algo:B2a}}{
      \For(){$n \in$ even sub-root list at node root $r$}{
        Apply DFS's to calculate node parities and delays (Equations \eqref{eq:nodeparity}, \eqref{eq:delayequation}) from $n$ to descendent nodes. Keep track of the largest value for $n_d$ encountered during the delay DFS.\;\label{algo:pdc}
      }
      Apply DFS from root $r$ to all descendents. At each node during the DFS, if $n_s=0$ (Equation \eqref{eq:suspension}), grow all boundary edges of vertices in the node a half-edge per the Union-Find decoder, such that grown edges are added to $\m{F}$, and apply $n_r=n_r+\nicefrac{1}{2}$. If $n_s\neq0$, apply $n_w=n_w+1$ and continue the DFS.\;\label{algo:grow}
    }
    \For(){edge $(u,v) \in \m{F}$\label{algo:B3a}}{
      \eIf(){$\codefunc{Find}(u)\neq\codefunc{Find}(v)$}{
        Merge vertex-trees by weighted \codefunc{Union}.\;
        \eIf(){$u \in n_v$ and $v \in n_v$\label{algo:joina}}{
          Merge node-trees by \codefunc{Join}.\;
        }(){
          Add $u$ to $n_v$ or $v$ to $n_u$.\;\label{algo:joinb}
        }
      }($u,v$ in same cluster.\label{algo:dfa}){
        Subtract 1 from $(u,v)$ in \emph{Support}.\;\label{algo:dfb}
      }
    }
    Update $\m{L}$ with odd clusters\; \label{algo:B3b}
  }
  Apply the peeling decoder \cite{delfosse2017linear}.\label{algo:B4a}
  \caption{Union-Find Node-Suspension decoder}\label{algo:ufbb}
\end{algorithm}
The full version of the algorithm we have described is given in Algorithm \ref{algo:ufbb}. Note that this pseudocode includes instructions that are shortened versions of the pseudocode of the Union-Find decoder \cite{delfosse2017almost}. This is done for clarity on the additions of the Node-Suspension data structure and protocols on top of the Union-Find pseudocode. The first block  of lines \ref{algo:B1a}-\ref{algo:B1b} initializes the clusters and describes the loop of cluster growth. Block 2 contains lines \ref{algo:B2a}-\ref{algo:grow} and describes the DFS's related to the calculation of node parities and delays from all even sub-roots stored at the root node, and the DFS of the cluster growth. Block 3 contains lines \ref{algo:B3a}-\ref{algo:B3b} and describes the combined merging protocols of the Union-Find and Node-Suspension data structures. Node that lines \ref{algo:dfa}-\ref{algo:dfb} contain an extra step to ensure that the vertex-trees are always acyclic. The final block in line \ref{algo:B4a} is the peeling decoder \cite{delfosse2017linear}, which now does not have to create the spanning forest of the grown clusters. Similarly to the Union-Find decoder, Weighted Growth is applied such that the smallers cluster is always grown first. 


\section{Complexity of Balanced-Bloom}\label{sec:complexity}

\Figure[htb](topskip=0pt, botskip=0pt, midskip=0pt){tikzfigs/tikz-figure6.pdf}{bla.).\label{fig6}}

In this section, we will find the worst-cast time complexity of the Union-Find Node-Suspension decoder. The addition cost of the original Union-Find decoder can be split in two parts: (A) the depth-first-searches (DFS's) related to the (re)calculation of the node parities and node delays in line \ref{algo:pdc}, and (B) the DFS related to the growth of a cluster in line \ref{algo:grow}. We dub these two parts the \textbf{suspension cost} and the \textbf{growth cost}, respectively. The \codefunc{Join} operation in lines \ref{algo:joina}-\ref{algo:joinb} only has a linear addition to the cost.

\subsection{Suspension cost}\label{sec:suscomplexity}

The cost of node suspension calculation is equal to the number of nodes traversed in the DFS's of the node parities and node delays. To find this number $N_{sus}(N)$ analytically, we utilize the rules of cluster growth and mergers.

\begin{enumerate}
  \item Weight growth states that clusters are grown in the order of sizes of their vertex-trees. 
  \item Joins between the node-trees between odd and even clusters always retains the root node of the odd cluster.
  \item The DFS's of the node parity and delay calculation are always performed just before cluster growth in an even partition of the node-tree starting from the pointer saved at the root node .
\end{enumerate}
From rules 2 and 3, we conclude that $N_{sus}$ is proportional to the sum of sizes of all even node-trees during all mergers of the growth process on the lattice. Note that if many clusters merge within the same growth iteration, only the last even cluster counts towards the cost, since rule 3 ensures that the calculation is not performed on intermediate even partitions. To find the worst-case time complexity, we maximize $N_{sus}$, which is proportional to the computation time. We take a top-down approach of \textbf{cluster fragmentation}; starting from a single cluster that maximally occupies the lattice at the end of growth, and move back in time to find its ancestor clusters and their sizes. The maximization of $N_{sus}$ is in the repetitiveness of the recalculation over some parition of the final node-tree. 

\begin{definition}\label{def:fragmentation}
  Let the \emph{fragmentation} of an odd cluster with node-tree $\pre{k-1}\nset^o$ split it into a set of its ancestral node-trees. Here $k$ indicates the \emph{generation}, where larger $k$ indicates a more distant ancestor set of smaller node-trees. Let the fragmentation step $f$ be the combination of \emph{partial fragmentations (PF)} $f_o$, which fragments an odd node-tree into an even ancestor and an odd ancestor
  \begin{equation}\label{eq:pfe}
    f_o(\{\pre{k-1}\nset^o\}) = \m{F}^o_k = \{\pre{k}\nset^e_{-1}, \pre{k}\nset^o_0 \}, 
  \end{equation}
  and PF $f_e$ fragments even node-trees into $n_f$ odd ancestors
  \begin{equation}\label{eq:pfo}
    f_e(\{\pre{k}\nset^e_{-1}\}) = \m{F}^e_k=\{\pre{k}\nset^{o}_1,...,\pre{k}\nset^o_{n_f}\},
  \end{equation}
  such that a fragmentation step is
  \begin{equation}\label{eq:fstep}
    f(\{\pre{k-1}\nset^o\}) = \m{F}_k = \{\pre{k}\nset^o_0,\pre{k}\nset^{o}_1,...,\pre{k}\nset^{o}_{n_f}\}.
  \end{equation}
\end{definition}

A fragmentation step is only possible on an odd node-tree $\nset^o$ with $|{\nset^o}| \geq 3$. The fragmentation functions does not remove single-node node-trees from a set when applied, such that for an odd cluster of finite size, after $k_m$ fragmentation steps all odd ancestors in $\m{F}_{k_{m}}$ are single nodes. Using fragmentations, we can sum over the fragmentation generations to find the sum of even node-tree sizes
\begin{equation}\label{eq:npdc}
  N_{sus} = 2\sum_{k=1}^{k_{m}}{ \sum_j{ \left\{ \abs{\pre{k}\nset_j^e} \big| \pre{k}\nset_j^e \in \m{F}^o_k(\pre{0}\nset^o)\right\} } },
\end{equation}
for which we require (a) the number of generations $k_m$, (b) the number of odd ancestors $n_f$ in Equation \eqref{eq:pfo} and (c) the ratio between the node-tree sizes of an odd node-tree and its even ancestor. To find these values, we make two assumptions to simplify the sum: 
\begin{enumerate}
  \item Assume that all odd ancestors at generation $k_m$ are of a single node. 
  \item Assume that cluster does not increase in size (vertex-tree) when grown, such that only mergers occur between existing clusters. 
\end{enumerate}
As a result of assumption 2, nodes are effectively not allowed to increase in radius. While this is only conceptually impossible, it maximizes the number of nodes on the lattice such that the sum in Equation \eqref{eq:pfo} forms an upper bound to $N_{sus}$. Furthermore, this assumption

\begin{theorem}\label{the:fragnumber}
  Fragmentation number $n_f=2$ maximizes $N_{sus}$ in equation \eqref{eq:npdc}.
\end{theorem}
\begin{proof}
  The sum of even node-tree sizes in each fragmentation step is constant per Lemma \ref{lem:equalevensum}. Thus, \eqref{eq:npdc} is maximized by having as many fragmentations steps as possible, or the largest possible $p$.  As $k_f$ increases the number of odd node-trees in each fragmentation step $f_o$, the average size of these odd node-trees has decreased. Consequently, the node-tree size decreases faster towards the minimum size of three nodes as more fragmentation steps are applied (Equation \eqref{eq:fstep}). As the sum of even node-tree sizes in each fragmentation step is the same, increasing $k_f$ decreases the number of fragmentation steps:
  \begin{equation}
    p \propto \frac{1}{k_f}.
  \end{equation}
  Hence, $N_{PDC}$ is maximized for the minimal value of $k_f$, which is $k_f = 2$.
\end{proof}


% \input{tikzfigs/fragmentation}

If partial fragmentation function $f_o$ is called on a set of node-trees $f_o(\{\nset^o, \nset^e, ...\})$, it fragments all odd node-trees in the set. Partial fragmentation function $f_e$ fragments all even node-trees. Along these lines, the entire set of odd node-trees $\m{F}_k$ can undergo the another fragmentation step into odd subsets, resulting in another set of ancestral node-trees $\m{F}_{k+1}$. We can do this some $p$ times on $\pre{0}\nset^o$, where we have selected $k-1=0$, until our resulting set of node-trees $\m{F}_{p}$ consists only of the smallest possible node subsets $\pre{p}\nset^o$ where $|\pre{p}\nset^o|=1$.


To find the worst-case complexity, we maximize $N_{PDC}$ or the cost of the partial calculations during the construction of the node-trees on the lattice. Let us assume the worst-case, when there are a maximal number of nodes in the node-trees just before the last round of growth. As the lattice is maximally occupied, this is a single odd node-tree $\pre{0}\nset^o$ in which a partial calculation is performed as part of the last round of growth. Node-tree $\pre{0}\nset^o$ has a maximal number of nodes if $|n.\vset|=1$ for all nodes $n$ in $\pre{0}\nset^o$. Thus, on a lattice of $N=|\vset|$ vertices, the node-tree $\pre{0}\nset^o$ has a maximal
\begin{equation}\label{eq:limitnsetsize}
  \abs{\pre{0}\nset^o} \leq N
\end{equation}
nodes. As the partial calculation is only executed on the even subtrees, $N_{PDC}$ is the sum of even node-trees sizes $|\pre{k}\nset^e|$, in all partial fragmentation sets $\m{F}^o_{k}$, during all fragmentation steps $k=[1,...,p]$, in the full fragmentation of $F(\pre{0}\nset^o)$. We add the factor 2 in Equation \eqref{eq:npdc} as both the parity calculation and delay calculations requires its own depth-first search. The sequence of fragmentations that maximizes the even node-tree sizes maximizes $N_{PDC}$.
% The worst-case delay complexity is computed by maximizing $N_{PDC}$ of the full fragmentation of $\pre{0}\nset^o$ with $S_{\pre{0}\nset^o} = N/2-1$.
\begin{definition}\label{def:fullfrag}
  Let the series of all $p$ fragmentation steps $f$ on $\pre{0}\nset^o$ be the \emph{full fragmentation} $F$, with
  \begin{equation}\label{eq:fullfrag}
    F(\pre{0}\nset^o) = \underbrace{f(f(...f(\pre{0}\nset^o)))}_\text{p times} = \{\pre{p}\nset^{o}_1, \pre{p}\nset^{o}_2,...,\pre{p}\nset^{o}_{N_\sigma} \} \hspace{.3cm} | \hspace{.3cm} \abs{\pre{p}\nset^{o}_i} = 1.
  \end{equation}
\end{definition}

\begin{definition}\label{lem:fragratio}
  Let the partial fragmentation ratio $R$ be the relative sizes of an ancestral node-tree $\pre{k+1}\nset$ and the fragmented node-tree $\pre{k}\nset$.
  \begin{equation}\label{eq:fragratio}
    zfragratio = \frac{\abs{\pre{k+1}\nset}}{\abs{\pre{k}\nset}}
  \end{equation}
\end{definition}
In $f_e$ there are a set of partial fragmentation ratios $\{R_{-1}, R_0\}$, and in $f_o$ are a set of partial fragmentation ratios $\{R_1,...,R_{k_f}\}$, where
\begin{align}
  R_{-1} +  R_0         & = 1  \\
  \sum_{i=1}^{k_f}{R_i} & = 1.
\end{align}

The problem of finding the sequence of even ancestral node-tree sizes to maximize the value of $N_{PDC}$ now becomes finding the partial fragmentation number $k_f$ and the set of partial fragmentation ratios $\{R_{-1},..., R_{k_f}\}$.

\begin{lemma}\label{lem:sumevenkf}
  For the same partial fragmentation ratios $\{R_{-1}, R_0\}$ in $f_o$, the sum of even ancestral node-tree sizes after a fragmentation step is not dependent on $k_f$ (see Figure \ref{fig:fragcorrect}).
\end{lemma}
\begin{proof}
  Let us consider an even node-tree $\pre{k}\nset^e$ that is first partially fragmented by $f_e$ to $\m{F}^e_k$. The fragmentation set $\m{F}^e_k$ is then partially fragmented by $f_o$ to $\m{F}_{k+1}^o$. Let us consider the two cases when $k_f=2$ and $k_f=4$. For $k_f=2$, the partial fragmentation $f_e$ splits $\pre{k}\nset^e$ into two odd ancestral node-trees in $\m{F}^e_k$ and four node-trees in $\m{F}_{k+1}^o$.
  % Let the size of $\pre{k-1}\nset^e$ be $|\pre{k-1}\nset^e| = K$. To find $n_o$, let us consider two cases where $n_o = 1$ or $n_o=2$. If an even node-tree $\pre{k-1}\nset^e$ is fragmented with $k_f=2$, a fragmentation step $f(\pre{k-1}\nset^e)=f_o(f_e(\pre{k-1}\nset^e))$ produces the following partial fragmentation sets:
  \begin{align*}
    % \nonumber % Remove numbering (before each equation)
    f_e(\pre{k}\nset^e)_{k_f = 2}
    = \m{F}^e_{k}|_{k_f = 2}
     & = \{ \pre{k} \nset^{o}_1, \pre{k} \nset^{o}_2\}                                                                             \\
    f_o(\m{F}^e_{k}|_{k_f = 2})
    = \m{F}^o_{k+1}|_{k_f = 2}
     & = \left\{ \{\pre{k+1}\nset^{o}_{0}, \pre{k+1}\nset^{e}\}^o_1 , \{\pre{k+1}\nset^{o}_{0}, \pre{k+1}\nset^{e} \}^o_2 \right\}
  \end{align*}
  The ratios of the sizes of fragmented node-trees in $f_e$ are
  \begin{equation*}
    \frac{\abs{\pre{k} \nset^{o}_1}}{\abs{\pre{k}\nset^e}} = R_1, \hspace{2em}
    \frac{\abs{\pre{k} \nset^{o}_2}}{\abs{\pre{k}\nset^e}} = R_2,
  \end{equation*}
  where $ R_1 + R_2 = 1$. The ratios of the sizes of fragmented node-trees in $f_o$ are
  \begin{equation*}
    \frac{\abs{\pre{k+1}\nset^{o}_0|^o_1}}{\abs{\pre{k} \nset^{o}_1}} =
    \frac{\abs{\pre{k+1}\nset^{o}_0|^o_2}}{\abs{\pre{k} \nset^{o}_2}} = R_0, \hspace{2em}
    \frac{\abs{\pre{k+1}\nset^{e}  |^o_1}}{\abs{\pre{k} \nset^{o}_1}} =
    \frac{\abs{\pre{k+1}\nset^{e}  |^o_2}}{\abs{\pre{k} \nset^{o}_2}} = R_{-1},
  \end{equation*}
  where $R_0 + R_{-1} = 1$. The sum of the sizes of even node-trees in the odd partial fragmentation set $\m{F}^o_{k+1}$ is thus
  \begin{equation*}
    R_1 R_{-1} \abs{\pre{k}\nset^e} + R_2 R_{-1} \abs{\pre{k}\nset^e} = (R_1 + R_2) R_{-1} \abs{\pre{k}\nset^e} = R_{-1} \abs{\pre{k}\nset^e}
  \end{equation*}

  For $k_f = 4$, the partial fragmentation sets are
  \begin{align*}
    % \nonumber % Remove numbering (before each equation)
    f_e(\pre{k}\nset^e)_{k_f = 4}
    = \m{F}^e_{k}|_{k_f = 4}
     & =\{ \pre{k}\nset^{o}_1, \pre{k}\nset^{o}_2,  \pre{k}\nset^{o}_3,\pre{k}\nset^{o}_4\}, \\
    f_o(\m{F}^e_{k}|_{k_f = 4})
    = \m{F}^o_{k+1} |_{k_f = 4}
     & = \big\{      \{ \pre{k+1}\nset^{o}_0, \pre{k+1}\nset^e\}^o_1,
    \{ \pre{k+1}\nset^{o}_0, \pre{k+1}\nset^e\}^o_2,                                         \\
     & \hspace{3em} \{ \pre{k+1}\nset^{o}_0, \pre{k+1}\nset^e\}^o_2,
    \{ \pre{k+1}\nset^{o}_0, \pre{k+1}\nset^e\}^o_4 \big\}.
  \end{align*}
  The ratios of the sizes of fragmented node-trees in $f_e$ are
  \begin{equation*}
    \frac{\abs{\pre{k} \nset^{o}_1}}{\abs{\pre{k}\nset^e}} = q_1, \hspace{2em}
    \frac{\abs{\pre{k} \nset^{o}_2}}{\abs{\pre{k}\nset^e}} = q_2, \hspace{2em}
    \frac{\abs{\pre{k} \nset^{o}_3}}{\abs{\pre{k}\nset^e}} = q_3, \hspace{2em}
    \frac{\abs{\pre{k} \nset^{o}_4}}{\abs{\pre{k}\nset^e}} = q_4,
  \end{equation*}
  where $ q_1 + q_2 + q_3 + q_4 = 1$. The ratios of the sizes of fragmented node-trees in $f_o$ are
  \begin{equation*}
    \frac{\abs{\pre{k+1}\nset^o_0|^o_1}}{\abs{\pre{k} \nset^{o}_1}} =
    \frac{\abs{\pre{k+1}\nset^o_0|^o_2}}{\abs{\pre{k} \nset^{o}_2}} =
    \frac{\abs{\pre{k+1}\nset^o_0|^o_3}}{\abs{\pre{k} \nset^{o}_3}} =
    \frac{\abs{\pre{k+1}\nset^o_0|^o_4}}{\abs{\pre{k} \nset^{o}_4}} = R_0,
  \end{equation*}
  and
  \begin{equation*}
    \frac{\abs{\pre{k+1}\nset^e|^o_1}}{\abs{\pre{k} \nset^{o}_1}} =
    \frac{\abs{\pre{k+1}\nset^e|^o_2}}{\abs{\pre{k} \nset^{o}_2}} =
    \frac{\abs{\pre{k+1}\nset^e|^o_3}}{\abs{\pre{k} \nset^{o}_3}} =
    \frac{\abs{\pre{k+1}\nset^e|^o_4}}{\abs{\pre{k} \nset^{o}_4}} = R_{-1},
  \end{equation*}
  where $R_0 + R_{-1} = 1$. The sum of the sizes of even node-trees in  the odd partial fragmentation set $\m{F}^o_{k+1}$ is thus
  \begin{align*}
    q_1 R_{-1} \abs{\pre{k}\nset^e} + q_2 R_{-1} \abs{\pre{k}\nset^e} + q_3 R_{-1} \abs{\pre{k}\nset^e} + q_4 R_{-1} \abs{\pre{k}\nset^e}
     & = (q_1 + q_2 + q_3 + q_4 ) R_{-1} \abs{\pre{k}\nset^e} \\
     & = R_{-1} \abs{\pre{k}\nset^e}.
  \end{align*}
  This is true for any $k_f = 2i, i\in \mathbb{N}^*$.
\end{proof}

\begin{lemma}\label{lem:equalevensum}
  The sum of even node-tree sizes in every fragmentation step $k$ is only dependent on partial fragmentation ratios $\{R_{-1}, R_0\}$.
  \begin{equation}\label{eq:equalevensum}
    \sum_j{ \left\{ \abs{\pre{k}\nset_j^e} \big| \pre{k}\nset_j^e \in \m{F}^o_k \right\} } = \text{constant}
    \hspace{1em} \bigg| \hspace{1em} \forall \m{F}_k^o \text{ during } F(\pre{0}\nset^o).
  \end{equation}
\end{lemma}
\begin{proof}
  Consider an odd node-tree $\pre{k-1}\nset^o$ that is partially fragmented as
  \begin{align*}
    f_o(\pre{k-1}\nset^o) = \m{F}^o_k     & = \{\pre{k}\nset^e_{-1}, \pre{k}\nset^o_0 \}                                                                                                                                 \\
    f_e(\m{F}^o_k)        = \m{F}^e_k     & = \left\{ \{\pre{k}\nset^o_i\ | i \in [1,..,k_f] \}^e_{-1}, \pre{k}\nset^{o}_0 \right\}                                                                                      \\
    f_o(\m{F}^e_k )       = \m{F}^o_{k+1} & = \left\{ \left\{ \{\pre{k+1}\nset^e_{-1}, \pre{k+1}\nset^o_0\}_i^o | i \in [1,..,k_f] \right\}^e_{-1}, \left\{\pre{k}\nset^e_{-1}, \pre{k}\nset^o_0 \right\}^{o}_0 \right\}
  \end{align*}

  The sum of even node-tree sizes in $\m{F}^o_k$ is simply the size of $\pre{k}\nset^e_{-1}$ and is equal to
  \begin{equation*}
    \sum_j{ \left\{ \abs{\pre{k}\nset_j^e} \big| \pre{k}\nset_j^e \in \m{F}^o_k \right\} } = \abs{\pre{k}\nset^e_{-1}} = R_{-1}\abs{\pre{k-1}\nset^o}.
  \end{equation*}

  The sum of even node-tree sizes in $\m{F}^o_{k+1}$ can be divided into two parts. The first part is the partial fragmentations $f_e f_o$ of $\pre{k}\nset^e_{-1}$, which we know from Lemma \ref{lem:sumevenkf} is $R_{-1}|\pre{k}\nset^e_{-1}|$ regardless of the choice for $k_f$. The second part is the partial fragmentation $f_o$ of $\pre{k}\nset^o_0$, which is $R_{-1}|\pre{k}\nset^o_0|$. Hence, the sum is
  \begin{equation*}
    \sum_j{ \left\{ \abs{\pre{k}\nset_j^e} \big| \pre{k}\nset_j^e \in \m{F}^o_{k+1} \right\} } = R_{-1} \left( \abs{\pre{k}\nset^e_{-1}} + \abs{\pre{k}\nset^o_0} \right) = R_{-1}\abs{\pre{k-1}\nset^o}.
  \end{equation*}
\end{proof}

\begin{theorem}\label{the:fragnumber}
  For the fragmentation number $k_f=2$, $N_{PDC}$ of Definition \ref{def:npdc} and Equation \eqref{eq:npdc} is maximized (see Figure \ref{fig:fragexamples}).
\end{theorem}
\begin{proof}
  The sum of even node-tree sizes in each fragmentation step is constant per Lemma \ref{lem:equalevensum}. Thus, \eqref{eq:npdc} is maximized by having as many fragmentations steps as possible, or the largest possible $p$.  As $k_f$ increases the number of odd node-trees in each fragmentation step $f_o$, the average size of these odd node-trees has decreased. Consequently, the node-tree size decreases faster towards the minimum size of three nodes as more fragmentation steps are applied (Equation \eqref{eq:fstep}). As the sum of even node-tree sizes in each fragmentation step is the same, increasing $k_f$ decreases the number of fragmentation steps:
  \begin{equation}
    p \propto \frac{1}{k_f}.
  \end{equation}
  Hence, $N_{PDC}$ is maximized for the minimal value of $k_f$, which is $k_f = 2$.
\end{proof}

% \input{tikzfigs/fragnumber}

The search for the fragmentation ratios has now been reduced to finding $\{R_{-1}, R_0\}$ of $f_o$ and $\{R_1, R_2\}$ of $f_e$ since $k_f = 2$. A partial fragmentation $f_e$ of $\pre{k}\nset^e_{-1}$ and fragmentation step $f$ of $\pre{k-1}\nset^o$ are now
\begin{align}
  f_e(\pre{k}\nset^e_{-1}) & = \m{F}^e_k  =\{\pre{k}\nset^{o}_1,\pre{k}\nset^o_2\} \label{eq:newpfe}                      \\
  f(\pre{k-1}\nset^o)      & = \m{F}_k    = \{\pre{k}\nset^o_0, \pre{k}\nset^o_1, \pre{k}\nset^o_1\}. \label{eq:newfstep}
\end{align}
The sizes of the ancestral odd node-trees in a fragmentation step $f$ are related to the joined node-tree by
\begin{equation}
  \frac{\abs{\pre{k}\nset^o_0}}{\abs{\pre{k-1}\nset^o}} = R_0, \hspace{2em}
  \frac{\abs{\pre{k}\nset^o_1}}{\abs{\pre{k-1}\nset^o}} = R_1, \hspace{2em}
  \frac{\abs{\pre{k}\nset^o_2}}{\abs{\pre{k-1}\nset^o}} = R_2,
\end{equation}
where
\begin{align}
  \nonumber  \tilde{R}_1                     & = R_{-1}R_1                      \\
  \tilde{R}_2                                & = R_{-1}R_2 \label{eq:bigratios} \\
  \nonumber  R_0 + \tilde{R}_1 + \tilde{R}_2 & = 1.
\end{align}

This fragmentation number does not come unexpectedly. If $k_f=2$, a fragmentation $f_e$ outputs two ancestral node-trees. This is equivalent to a single even-join. If $k_f>2$, the fragmentation $f_e$ will be equivalent to several even-joins and intermediate odd-joins. Recall from Lemma \ref{lem:delaywhengrown} that the partial calculation of every intermediate odd-join is skipped. Thus, these partial calculations are ``lost'' from the maximization of $N_{PDC}$.

Let us now try to maximize $N_{PDC}$ of Equation \eqref{eq:npdc}, not from the perspective of fragmentations, but from the perspective of cluster growth. During a growth iteration, some $N_v$ vertices are added to the cluster $c_j$, and some other clusters merge with $c_j$ that also require the join of their respective node-trees. If no join operations occur, the node-tree stays unchanged. The cluster is allowed to continue to grow without delay calculations per Lemma \ref{lem:calconce}. To maximize $N_{PDC}$, $N_v$ must be minimized, as every added vertex here is one that could have been part of a node in some other node-tree, and thus does not add to $N_{PDC}$.


\begin{lemma}\label{lem:localmax}
  Local maximization of the fragmentation ratio of the even subtree $R_{-1}=\tilde{R}_1+\tilde{R}_2$ leads to global maximization of $N_{PDC}$.
\end{lemma}
\begin{proof}
  Recall from Equation \eqref{eq:pfe} that in the partial fragmentation $f_o$ of $\pre{k-1}\nset^o$, the size of the fragmented even ancestral subtree is
  \begin{equation*}
    \abs{\pre{k}\nset^e_{-1}} = R_{-1}\abs{\pre{k-1}\nset^o},
  \end{equation*}
  which counts towards $N_{PDC}$ in the full fragmentation. In the fragmentation step $k$, $\pre{k-1}\nset^o$ is fragmented according to fragmentation ratios $\{R_0, \tilde{R}_1, \tilde{R}_2\}_k$. In the next fragmentation step $k+1$, the odd node-trees of $\m{F}_k=\{\pre{k}\nset^o_0, \pre{k}\nset^o_1, \pre{k}\nset^o_1\}$ are to fragmented into ancestral node-trees. The framentation ratios for each of the fragmentations $f(\pre{k}\nset^o_0), f(\pre{k}\nset^o_1), f(\pre{k}\nset^o_1)$ are not dependant on $\{R_0, \tilde{R}_1, \tilde{R}_2\}_k$. Thus global maximization of $N_{PDC}$ is achieved by local maximization of $R_{-1}=\tilde{R}_1+\tilde{R}_2$ in every fragmentation.
\end{proof}

\begin{theorem}\label{the:fragratio}
  For the fragmentation ratios $R_0 = \tilde{R}_1 = \tilde{R}_2 = \frac{1}{3}$, $N_{PDC}$ of Definition \ref{def:npdc} and Equation \eqref{eq:npdc} is maximized in a Union-Find Balanced-Bloom decoder with weighted growth (see Section \ref{sec:bucketwg}).
\end{theorem}
\begin{proof}
  Take the partial fragmentation $f_o$ of $\pre{k-1}\nset^o$ of Equation \eqref{eq:pfo} and $f_e$ of Equation \eqref{eq:newpfe}, which are equivalent to a final odd-join between $\nset^e_{-1}, \nset^o_0$ and even-join between $\nset^o_1, \nset^o_2$, respectively.

  For $f_e$ that is equivalent to the even-join to even-parity cluster $c_{-1}$ between the odd-parity clusters $c_1, c_2$ with node-trees $\nset^o_1, \nset^o_2$, the clusters must have relatively equal vertex-tree sizes
  \begin{equation*}
    \abs{\vset_1} \approx \abs{\vset_2}.
  \end{equation*}
  If not, clusters $c_1, c_2$ may be allowed to grow multiple iterations before merging, sorted by weighted growth. In each iteration, some $N_v$ vertices are added to the cluster. Let the growth iteration in which the even-join occurred be labeled as $i_e$

  For $f_o$ equivalent to the final odd-join between even $c_{-1}$ and odd $c_0$, the final odd-join must strictly occur after the even-join of $f_e$. This odd-join can either be initiated by odd-parity $c_0$ in some growth iteration $i_o > i_e$, when $c_2$ is the only odd-parity cluster, or initiated by either $c_1$ or $c_2$ in growth iteration $i_o = i_e$ (see Figure \ref{fig:fragfratio}). Determined by weighted growth, the vertex-tree sizes are related as
  \begin{equation*}
    \abs{\vset_0} \geq \abs{\vset_1} \approx \abs{\vset_2}.
  \end{equation*}
  Recall from equation \eqref{eq:sets} that $|\nset| \leq |\vset|$. We assume the largest possible node-tree size $|\nset| = |\vset|$ to find that
  \begin{equation*}
    \abs{\nset^o_0} \geq \abs{\nset^o_1} \approx \abs{\nset^o_2},
  \end{equation*}
  and hence
  \begin{equation*}
    R_0 \geq \tilde{R}1 \approx \tilde{R}2,
  \end{equation*}
  To maximize $N_{PDC}$, we want to maximize $\abs{\nset^e_{-1}} = \abs{\nset^o_1} + \abs{\nset^o_2}$ or $R_{-1}=\tilde{R}1 + \tilde{R}2$ (per Lemma \ref{lem:localmax}\footnote{Added in revision}). Since \eqref{eq:bigratios}, $R_0$ must be as small as possible, and thus $R_0 = \tilde{R}1 = \tilde{R}2 = \frac{1}{3}$.
\end{proof}

% \input{tikzfigs/fragratio}

The last unknown parameter for the maximization of $N_{PDC}$ in Equation \eqref{eq:npdc} is $p$, the total number of fragmentation steps. If we assume that in every growth step, not a single non-node vertex is added $N_v = 0$, the full fragmentation of odd node-tree $\pre{0}\nset^o$ is just the continuous division of the set in 3 parts per Theorem \ref{lem:fragratio}, which can be calculated easily.
\begin{equation}\label{eq:numfrag}
  p \leq \log_3(\abs{\pre{0}\nset^o})
\end{equation}
In every partial fragmentation set $\m{F}^o_k$, the sum of even node-tree sizes is
\begin{equation}\label{eq:sumevensetsize}
  \sum_j \left\{ \abs{\pre{k}\nset_j^e} \big| \pre{k}\nset_j^e \in \m{F}^o_k \right\} \leq \frac{2}{3}\abs{\pre{0}\nset^o},
\end{equation}
as $R_1+R_2 = \frac{2}{3}$ per Theorem \ref{the:fragratio}, and this value is constant for every fragmentation step per Lemma \ref{lem:equalevensum}. This is an inequality as we have assumed $|\nset| = |\vset|$ and $N_v=0$ in Theorem \ref{the:fragratio}. Filling in equation \eqref{eq:numfrag} and \eqref{eq:sumevensetsize} in \eqref{eq:npdc}, we find that
\begin{align}
  % \nonumber % Remove numbering (before each equation)
  \nonumber N_{PDC} & \leq \sum_{k=1}^{p} \sum_j \left\{ \abs{\pre{k}\nset_j^e} \big| \pre{k}\nset_j^e \in \m{F}^o_k \right\} \\
  \nonumber         & \leq \sum_{k=1}^{\log_3(\abs{\pre{0}\nset^o})} \frac{2}{3}\abs{\pre{0}\nset^o}                          \\
                    & \leq \frac{2}{3}\abs{\pre{0}\nset^o}\log{\abs{\pre{0}\nset^o}}.
\end{align}

Recall from Equation \eqref{eq:limitnsetsize} that the node-tree size is bounded by the lattice size $|\pre{0}\nset^o| \leq N$. The worst-case time complexity of the delay computation is thus bounded by $\m{O}(N\log{N})$. The average-case complexity is even lower as it is almost certain that not all vertices are nodes such that $|\nset| < |\vset|$ and $N_v \neq 0$.

\subsection{Bloom complexity}\label{sec:bloomcomplexity}

To grow a cluster represented by a node-tree $\nset$, a depth-first search is performed on the node-tree to iterate over each boundary list that are stored at the nodes.
\begin{definition}\label{def:nbloom}
  Let the total number of times nodes are bloomed with \codefunc{Bloom} be $N_{bloom}$.
\end{definition}

Similar to the previous section, we assume a maximum number of nodes on the lattice where, in each cluster $|\nset| = |\vset|$ and $N_v = 0$. For a fragmentation step of $f(\pre{k-1}\nset^o)$ to $\{\pre{k}\nset^o_0, \pre{k}\nset^o_1, \pre{k}\nset^o_2\}$, $N_{bloom}$ is maximized if all three ancestral node-trees are grown. As the growth of every set $\nset$ adds $|\nset|$ to $N_{bloom}$, the total number of bloom can be found similarly to $N_{PDC}$ in Equation \eqref{eq:npdc}. The sum is now on all odd node-tree sizes in all $p$ fragmentation steps $\m{F}_k$:
\begin{equation}\label{eq:nnode}
  N_{bloom} \leq \sum^{p}_{k=1}\sum_j \left\{ \abs{\pre{k}\nset^o_j} \big| \pre{k}\nset^o_j \in \m{F}_k \right\}
\end{equation}
For a full fragmentation of $\nset$ of size $|\nset|$, the sum of all set sizes in each fragmentation set $\m{F}$ is
\begin{equation}\label{eq:sumsetsfrag}
  \sum_j \left\{ \abs{\pre{k}\nset_j} \big| \pre{k}\nset^o_j \in \m{F}_k \right\} = \abs{\pre{0}\nset^o}.
\end{equation}
By filling in $p$ from \eqref{eq:numfrag}, we find that
\begin{align}
  % \nonumber % Remove numbering (before each equation)
  \nonumber & N_{bloom} & \leq \sum^{p}_{k=1}\sum_j \left\{ \abs{\pre{k}\nset^o_j} \big| \pre{k}\nset^o_j \in \m{F}_k \right\} & \\
  \nonumber &           & \leq \sum_{k=1}^{\log_3{\abs{\pre{0}\nset^o}}} \abs{\pre{0}\nset^o}                                  & \\
            &           & \leq \abs{\pre{0}\nset^o}\log_3{\abs{\pre{0}\nset^o}},                                               &
\end{align}
which again corresponds to a worst-case time complexity that is bounded by $\m{O}(N\log{N})$.

\section{Performance}\label{sec:performance}


\Figure[htb](topskip=0pt, botskip=0pt, midskip=0pt){tikzfigs/threshold_ufbb.pdf}{bla\label{threshold_ufbb}}
\Figure[htb](topskip=0pt, botskip=0pt, midskip=0pt){tikzfigs/comp_ufbb_toric_2d_p98.pdf}{bla\label{tmw_comp}}
\Figure[htb](topskip=0pt, botskip=0pt, midskip=0pt){tikzfigs/threshold_comparison.pdf}{bla\label{thres_comp}}
% \Figure[htb](topskip=0pt, botskip=0pt, midskip=0pt){tikzfigs/threshold_comparison_dense.pdf}{bla\label{thres_comp_d}}
% this can be plotted with a shared y-axis
\section{Conclusion}\label{sec:conclusion}

In this thesis, we have thoroughly inspected the Union-Find decoder that has the advantage in running time and simplicity compared to other decoders. We have provided various implementations of the original Union-Find decoder with the bucket sort method for weighed growth and the dynamic forest variant of the decoder. We have introduced a modification of the original decoder that selectively grows regions of clusters based on the concept of a potential matching weight. The modified decoder, dubbed the Union-Find Balanced-Bloom (UFBB) decoder, relies on an additional node-tree data structure to facilitate the calculation of the potential matching weight. A direct comparison of the decoding performance of our decoder, the Union-Find decoder (with original data \cite{delfosse2017almost}), and the Minimum-Weight Perfect Decoder in Figure \ref{fig:directcomp}. We summarize our results in short:

\begin{itemize}
  % \item By maintaining a dynamic forest, the threshold of the Union-Find decoder can be slightly increased. 
  \item The UFBB decoder has an improved decoder threshold compared with the Union-Find decoder for all lattice sizes and a comparable threshold to the Minimum-Weight Perfect Matching decoder for small lattices.
  \item The UFBB decoder has an improved decoding rate at the threshold error rate compared with the Union-Find decoder for all lattice sizes. 
  % \item A degradation of the error threshold of the UFBB decoder on larger lattice sizes is expected due to the effects of parity inversion and increased node delays. 
  % \item The matching weight of the UFBB decoder is decreased from the matching weight of the Dynamic-forest Bucket Union-Find decoder by constant factor, depending on the error model. 
  % \item For a constant value of the equilibrium factor, $k_{eq}=\frac{1}{2}$ results in the best performance in the UFBB decoder. 
  \item The UFBB decoder has a worst-case time complexity of $\m{O}(N\log{N})$. 
\end{itemize}

Many of the work on recent decoders have focused on increasing the code threshold via, for example, the use of neural networks. However, these decoders have the disadvantage that they have a significant running time and bad scalability. The Union-Find decoder manages to decode fast and scale almost-linearly with the input system size. For this reason it may be a great candidate for physical applications in the near future. We manage to find a middle ground between the two objectives as the Union-Find Balanced-Bloom decoder has an improved decoding performance at the cost of a slight increase in complexity. 

% Recent work that includes the Union-Find decoder focuses on bringing the decoder algorithm to the hardware level. Most notably, a scalable decoder micro-architecture has been proposed with a fully pipelined hardware implementation \cite{das2020scalable}. Related work has shown that a reduction in bandwidth is possible provided qubits with a low physical error rate \cite{delfosse2020hierarchical}. Furthermore, another variant of the decoder dubbed the \emph{Weighted Union-Find} decoder, not to be confused with \emph{weighted growth}, promises to increase the code threshold under circuit-level noise \cite{huang2020fault}. This application relies on adopting the decoder to a \emph{weighted} graph. Every edge $e\in\m{E}$ may now have a different length value, and edges are not limited to the growth of half-edges per growth iteration. We believe that Union-Find Balanced-Bloom decoder and the Weighted Union-Find decoder are compatible. In the combined decoder, boundary edges in every node are grown with respect to their weights in the weighted graph. For these reasons, the Union-Find decoder has a bright future in the world of quantum algorithms.


\FloatBarrier
\printbibliography
\EOD
\end{document}