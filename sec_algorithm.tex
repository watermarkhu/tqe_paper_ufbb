\section{Union-Find Partitioned-Growth decoder}\label{sec:ufbb}
We describe the \emph{Union-Find Partitioned-Growth} decoder in this section. We first introduce the concept of the potential matching weight in \Cref{sec:matchingweight}. The Union-Find Partitioned-Growth decoder uses the potential matching weight as a heuristic to prioritize partitioned growth of clusters. We introduce the node-tree data structure required for this decoder in \Cref{sec:nodeset}, and describe how to approximate the potential matching weight leveraging the node-tree in \Cref{sec:paritydelay}. In \Cref{sec:grownodetrees,sec:nodejoin,sec:inversion}, we show how to approximate efficiently the potential matching weight. The algorithm is described in \Cref{sec:pseudocode}. 

\subsection{Potential Matching Weight}\label{sec:matchingweight}

\Figure[bt](topskip=0pt, botskip=0pt, midskip=0pt){figures/tikz/build/main-figure0.pdf}{
    A cluster with vertices $\{a,b,c\}$ with potential matching weights $\{2\frac{1}{2}, 3\frac{1}{2}, 2\frac{1}{2}\}$. The line style and color of the colored edges correspond to the matching in the hypothetical union with an external vertex $v'$ of the same line style and color.\label{fig0}}

%In the following we give some intuition into the improvement of the Union-Find Balanced Bloom decoder upon the original Union-Find decoder. 
% We compared the ratio of the matchings between the MWPM decoder and our own implementation of the UF decoder, averaged over many simulations, and found that UF matching weight has a constant prefactor of $\sim 1.043$ over the minimum weight for the toric code (\Cref{comp_weight}). From this, we suspected that a decreased matching weight is a heuristic for an increased threshold. Within the context of the UF decoder, the matching weight may be decreased by prioritizing the growth of vertices with low PWM's within the cluster. 

Consider a cluster containing the set of an odd number of non-trivial vertices $V=\{a,b,c\}$ and the set of edges $E=\{(a,b), (b, c)\}$ of \Cref{fig0}. Now let us investigate the weight of a matching if an additional non-trivial vertex $w$ is connected to the cluster. If $w$ is connected to $a$ or to $c$, then the resulting matching has a total weight of 2: $(w,a)$ and $(b,c)$, or $(a,b)$ and $(c,w)$. However, if $w$ is connected to vertex $b$, then the total weight is 3: $(w, b)$ and $(a, c)$. %If we want to minimize the weight of the 

Inspired by this observation, we associate with each vertex $v$ of an odd-parity cluster a \textbf{Potential Matching Weight} $\text{PMW}(v)$. The PMW measures the matching weight, assuming a union occurs in the next growth iteration. More precisely,
%\begin{definition}
    % define cluster
    % define matching size or weight
let $v$ be a vertex in the boundary of an odd cluster and let $w$ be a hypothetical non-trivial vertex adjacent to $v$ but exterior to the cluster. We define $\text{PMW}(v)$ as the minimum-weight perfect-matching's weight of the even cluster that results from adding the $(v,w)$ vertex to the odd cluster. % is given by the 
    %Let there be a hypothetical merger between odd cluster $\alpha$ of vertices $V_\alpha$ and edges $E_\alpha$, and odd cluster $\beta$ of $V_\beta$ and $E_\beta$, on the edge $(v_\alpha, v_\beta)$, where $v_\alpha \in V_\alpha$ and $v_\beta \in V_\beta$. In the merged even cluster with edges $E_{\gamma} = E_\alpha \cup E_\beta \cup (v_\alpha, v_\beta)$, there is a matching $\m{C}_{(v_\alpha,v_\beta)} \subseteq E_{\gamma}$  between the syndrome vertices internal to the cluster. The \textbf{Potential Matching Weight} (PMW) of vertex $v_\alpha$ is then defined as
    %\begin{equation}
      %\text{PMW}(v) = \abs{\m{C}_{(v_\alpha,v_\beta)} \cap E_\alpha} + 1.
     % \text{PMW}(v) = \abs{\m{C'}}.
    %\end{equation}
    %\end{definition}
    
Since per growth iteration half-edges are attached to the odd cluster's boundary, the addition of an edge $(v,w)$ to a cluster occurs in two steps. In the first step, half of $(v,w)$ is added to the cluster of $v$. If $w$ is part of another cluster, and if the other half of $(v,w)$ was already added to the cluster of $w$, $(v,w)$ becomes \emph{fully grown}. If this is not the case, another round of growth is required to fully grow $(v,w)$ and to merge the clusters of $v$ and $w$. The PWM of $v$ is identical in both cases. To distinguish between the two cases, we add $\nicefrac{1}{2}$ to the PMW's of vertices with half-edges attached. Using this definition, the vertices $\{a,b,c\}$ in the cluster shown in \Cref{fig0} have PMW's $\{2\nicefrac{1}{2}, 3\nicefrac{1}{2}, 2\nicefrac{1}{2}\}$. The PMW can be used to prioritize the growth of vertices with low PMW such that there is an increased probability of mergers between clusters on edges connected to these vertices, and there is an increased probability of a lower matching weight. However, this heuristic is only interesting if the calculation of the PMWs is lighter computationally than performing minimum-weight perfect-matching. %within a cluster is potentially , especially for clusters of increasingly larger size, as all edges of a cluster must be considered in its calculation. Furthermore, the PMWs within a cluster change due to cluster growth and mergers, both of which occur more frequently as the system size increases. For this reason, the scaling of the PMW computation is vital to the decoder. 


\Figure[bt](topskip=0pt, botskip=0pt, midskip=0pt){figures/tikz/build/main-figure1.pdf}{
    The cluster of \Cref{fig0}, comprised of nodes $\{A, B, C\}$ with respective roots $\{a, b, c\}$, after two rounds of prioritized growth of $a$ and $c$. There are regions of vertices that are either interior elements or have equal potential matching weights, represented as nodes with different node radii (labelled below the node) in the node-tree $\nset$. \label{fig:pmw}}

\subsection{The node tree}\label{sec:nodeset}
Fortunately, the PMW can be approximated efficiently. 
For this we introduce the \textbf{node-tree} data structure for a cluster $C$. 

A node-tree $\nset$ (see \Cref{fig:nodes}) is a contraction of a spanning tree of $C$. Each node $N\in\nset$ represents a sub-tree that contains at most one non-trivial vertex and whose leaf vertices lie at equal distance from the sub-tree root, the \textbf{node radius} $\rho_N$. In the following we explain how to choose the nodes such that the leaves of the sub-tree represented by a node have equal PMW.

We define two types of nodes. First, we let every non-trivial vertex be the root of a node, we call these nodes \textbf{syndrome-nodes}. The nodes can grow separately from each other according to different strategies, but we assume that they grow uniformly, i.e. when a node grows, all leaves are grown by half an edge.

As the nodes grow, they can meet with each other. We call a \textbf{junction} a vertex in the boundary to two or more nodes. When two nodes meet at a junction that is equidistant to them, we define a second type of node, a \textbf{junction-node}, with root the junction vertex. 

%Consider the cluster of non-trivial vertices $V_i=\{v_0,v_1,v_2\}$ and edges $E_i = \{(v_0,v_1), (v_1, v_2)\}$ from \Cref{fig0}. We had found previously that vertices $v_0, v_2$ have a lower PMW compared to $v_1$ by 1 edge. The growth of $v_0$ and $v_2$ could be prioritized, such that new vertices are added to the cluster on the boundary of $v_0$ and $v_2$. If all newly added vertices are trivial, the cluster would be described by \Cref{fig:pmw}. If we repeat the PMW calculation, we now find that the PMWs of the new vertices connected to $v_0$ are equal. The same holds for the vertices connected to $v_2$.

We denote syndrome and junction nodes by an upper case letter decorated with a dot, $\dot{N}$, or a bar, $\bar{N}$, respectively. Given a set of nodes $\nset$, we define its parity as the parity of the number of syndrome-nodes. We decorate odd and even sets of nodes with a dot, $\dot\nset$, or a bar, $\bar\nset$, respectively.

%Note that similarly to trivial vertices on the cluster, junction-nodes do not count towards the parity of the node-tree. 

Node trees are a convenient data structure for computing the PMW. % for several reasons. %computing the PMW. 
%First, the weight of a matching in $\vset$ is equal to the weight of the same matching in $\nset$. 
They avoid redundant computations since all of the leaves in each node have equal PMW. For the example in \Cref{fig:pmw}, the PMW of the leaves of $A$, henceforth just the PMW of $A$, is $\rho_A + \abs{(B,C)} +1$. Moreover, this succinct representation is easy to maintain: a growth iteration can be modeled by an increase in the node radius by $\nicefrac{1}{2}$.

\Figure[bt](topskip=0pt, botskip=0pt, midskip=0pt){figures/tikz/build/main-figure2.pdf}{
Two different node types. Syndrome-nodes $\dot N$ have a non-trivial vertex at its center. Vertices that lie on the radii of two existing nodes initiate a junction-node $\bar{N}$ in the node-tree. Vertex $b$ lies on equal distances to root vertices $a$ of node $\dot{A}$ and $c$ of $\dot{C}$, and vertex $d$ to root vertices $c$ of $\dot{C}$ and $e$ of $\dot{E}$. Therefore, vertices $b$ and $d$ initiate junction-nodes $\bar{B}$ and $\bar{D}$, respectively, in the merged node-tree. \label{fig:nodes}}

We note that the node tree needs to be maintained in addition to the cluster trees. The role of the cluster trees is to differentiate between clusters.
%structure does not replace but coexists with the Union-Find data structure, whose \textbf{cluster-trees}' critical goal is to differentiate between clusters \cite{delfosse2017almost}. 
We depict the different relevant data structures %The relevant data structures and the cluster-tree, spanning-tree $\vset$ and node-tree $\nset$ 
of an example cluster in \Cref{fig:nodetypes}. %Note that the set of all spanning-trees $\{\vset\}$ is equivalent to the spanning-forest $F$, which is constructed in the original Union-Find decoder after cluster growth and before peeling. When all clusters are of even parity, the set of spanning-trees $\{\vset\}$ can be passed to the peeling decoder \cite{delfosse2017linear}. 

\subsection{Node parity and delay}\label{sec:paritydelay}
%The Partitioned-Growth data structure allows for prioritizing the growth of \emph{nodes} on an odd-parity node-tree $\dot{\nset}$ with low PWM, instead of \emph{vertices}. 
The PMW can be used to prioritize the growth of vertices with low PMW. However, in order to implement this heuristic, the relevant figure of merit is not the PMW itself but the difference in PMW between the vertices. In the following we explain how to track the differences in PMW efficiently.

%%%%%%%%%%%%%%%%%%%
%We define the delay of a node $N$ which we denote by $\delta_N$, as the difference between the PMW of $N$ and the minimum PMW in the node-tree multiplied by two. 
%\begin{equation}\label{eq:delayequation}
%    \delta_N = 2\left(\text{PMW}(N) - \min_{X\in\nset}{\text{PMW}(X)}\right).
%\end{equation}
%The delay of a node can be interpreted operationally as the number of times the node with minimum PMW should grow to have the same PMW as node $N$. %rowth iterations for a node to wait for all nodes in the node-tree to have equal PWM: 

%To grow a cluster, we loop over all nodes in $\nset$ and grow all boundary vertices of $N$ if $\delta_N=0$. 
%However, to find $\delta_N$, the PMW of all nodes in $\dot{\nset}$ must be known, which is no trivial task as the entire tree must be considered for the calculation in every node. Furthermore, the PWM of a node changes as it and its cluster grow. Instead, we will compute for the difference in node delay between child node and its parent. 
%%%%%%%%%%%%%%%%%%%%%

Let a node $R\in \nset$ be the root node of the node tree. As $\nset$ is an acyclic tree, $R$ defines implicitly the parent-child relations for all the nodes in $\nset$, where a node has one parent, but potentially multiple children. We let $P\geq Q$ denote that $P$ is the parent of $Q$. Abusing notation, we represent any odd-parity node-tree of $\abs{\dot\nset}\geq 3$ by the tuple $\{\nset_P,P,Q,\nset_Q\}$, where $\nset_P$ denotes the sub-tree consisting of all ancestors of $P$ and all descendant branches excluding the branch starting from $(P,Q)$, and $\nset_Q$ denotes the sub-tree consisting of all descendant branches of $Q$. With this notation  %, where $(\nset_P, P)$ and $(Q, \nset_Q)$ can represent any number of edges on $\nset$, depending on the number branches they represent. 
and based on the node types of $P,Q$ and the parity of $\nset_Q$, we can deduce the parity of $\nset_P$, since $\dot{\nset}$ is odd. 

In the following, we derive the difference in PMW between a parent node $P$ and its child $Q$. For this, let us consider that a syndrome node is connected to either $P$ or $Q$. 

We first argue that any matching is equivalent to a matching where all nodes in the even node are matched internally. If the sub-tree $\nset_Q$ is of even parity, denoted by $\overbar{\nset_Q}$, it must consist of a combination of even branches and an even number odd branches, where a branch is the subtree hanging from a child node. %, where the parity of the branch refers to the parity of the number of syndrome-nodes. 
For any even branch $\bar{b}$, the full matching must be within $\bar{b}$: since there is a single edge connecting $\bar b$ with $Q$ and the branch is even, it would not be possible to support two matches out of $\bar{b}$. 
If an odd branch $\dot{b}_1$ matches outside $\overbar{\nset_Q}$, on edge $(Q, \dot{b}_1)$, there must exist another odd branch $\dot{b}_2$ that also matches outside $\overbar{\nset_Q}$, in this case via edge $(Q, \dot{b}_1)$. Since the matchings meet at node $Q$ there is an equivalent matching that matches $\dot{b}_1$ and $\dot{b}_2$. 
%If the node-tree merges with another odd tree either at $Q$ or $P$, the minimum weight matching will be equivalent to a matching where all nodes in $\overbar{\nset_Q}$ match internally. 
The same argument holds for an even $\nset_P$.

\textcolor{red}{DEFINE MATCHING and WEIGHT}

If the subtree $\nset_Q$ is of odd parity, denoted by $\dot{\nset_Q}$, it must consist of a combination of even branches and an odd number of odd branches. By the same argument as before, for any matching, there exists an equivalent matching where the even branches and all odd branches except for one (which is an even number) match internally. 
The remaining odd branch in $\dot{\nset_Q}$ must be connected with either $Q$, $P$, or the new syndrome-node connected. 
The same argument holds for an odd $\nset_P$.

Using this argument, we can deduce the required edges for a matching during a merger on $Q$ and $P$ and find the difference in the PMW of $Q$ and $P$. For example, if both $P,Q$ are syndrome-nodes and $\nset_Q$ is even, $\nset_P$ must be odd. The potential matching weights on $\dot{P},\dot{Q}$ are thus
\begin{align*}
  \text{PMW}(\dot{P}) &= \rho_{\dot{P}} + \abs{(\dot{\nset_P},\dot{P})} + \abs{(\dot{P}, \dot{Q})} + C \\
  \text{PMW}(\dot{Q}) &= \rho_{\dot{Q}} + \abs{(\dot{\nset_P},\dot{P})} + C,
\end{align*}
where $C$ is a constant equal to the weight of the matching with the subtrees $\dot{\nset_P}$ and $\overbar{\nset_Q}$ and \textcolor{red}{$\abs{(\dot{\nset_P},\dot{P})}$ denotes the distance between the roots of $\nset_P$ and $P$}. In consequence, we find that their difference is 
\begin{equation*}
  \text{PMW}(\dot{Q}) - \text{PMW}(\dot{P}) = \rho_{\dot{Q}} - \rho_{\dot{P}} - \abs{(\dot{P}, \dot{Q})}.
\end{equation*}

Similarly, the PMW difference can be deduced for any combination of node types of $P,Q$ and parity of $\nset_Q$ (see \Cref{fig:nmcombi}):
\begin{align*}
  \text{PMW}({Q}) - \text{PMW}({P}) &= \rho_Q - \rho_P - \abs{(P,Q)}  &\dot{P}, \dot{Q}, \overbar{\nset_Q} \\
  &= \rho_Q - \rho_P + \abs{(P,Q)}  &\dot{P}, \dot{Q}, \dot{\nset_Q} \\
  &= \rho_Q - \rho_P + \abs{(P,Q)}  &\dot{P}, \bar{Q}, \overbar{\nset_Q} \\
  &= \rho_Q - \rho_P - \abs{(P,Q)}  &\dot{P}, \bar{Q}, \dot{\nset_Q} \\
  &= \rho_Q - \rho_P - \abs{(P,Q)}  &\bar{P}, \dot{Q}, \overbar{\nset_Q} \\
  &= \rho_Q - \rho_P + \abs{(P,Q)}  &\bar{P}, \dot{Q}, \dot{\nset_Q} \\
  &= \rho_Q - \rho_P + \abs{(P,Q)}  &\bar{P}, \bar{Q}, \overbar{\nset_Q} \\
  &= \rho_Q - \rho_P - \abs{(P,Q)}  &\bar{P}, \bar{Q}, \dot{\nset_Q} \\
\end{align*}
which can be simplified to 
    %n_d = m_d + \Big\lfloor 2C\big(n_r-\rho_P - (-1)^{n_p}\abs{(n,m)}\big )\Big\rfloor
\begin{multline}\label{eq:pmwdif}
    \text{PMW}({Q}) - \text{PMW}({P}) %= \\
    =\rho_Q-\rho_P + (-1)^{\pi_Q}\abs{(P,Q)},% \hspace{.3cm} Q\leq P,
\end{multline}
where $\pi_N$ is the \textbf{node parity} of $N$, i.e. the number of syndrome-nodes modulo 2. The node parity of a node can also be calculated recursively from the node parities of its children nodes. In particular, they are given by the following expressions for syndrome-nodes and junction-nodes: 
\begin{align}
    \pi_{\dot{P}} &= 1+\left(\hspace{.2cm} \sum_{\mathclap{\forall Q \leq \dot{P}}} (1+\pi_Q) \right ) \bmod 2, \label{eq:snodeparity}\\
    \pi_{\bar{P}} &= \hspace{.6cm} \left(\hspace{.2cm} \sum_{\mathclap{\forall Q \leq \bar{P}}} (1+\pi_Q) \right) \bmod 2. \label{eq:jnodeparity}
\end{align}

\Figure[tb](topskip=0pt, botskip=0pt, midskip=0pt){figures/tikz/build/main-figure3.pdf}{
  Any odd-parity node-tree of $\abs{\dot{\nset}}\geq 3$ can be simplified to $\{\nset_P,P,Q,\nset_Q\}$, where $\nset_P$ is the subtree consisting of all ancestors of $P$ and all descendant branches excluding the branch starting from $(P,Q)$, $\nset_Q$ is the subtree consisting of all descendant branches of $Q$, and $(\nset_P, P), (Q, \nset_Q)$ can represent any number of edges on $\dot{\nset}$. The figure shows which edges are part of the matching for a hypothetical merger on either $P$ (cyan edges) or $Q$ (magenta edges), for any combination of node types of $P,Q$ and the parity of $\nset_Q$, similarly to \Cref{eq:pmwdif}.
  \label{fig:nmcombi}}

The previous relations allow to compute the difference in PMW between nodes directly connected in the node-tree. These relations can be used to obtain the difference with respect to a reference node, for instance with respect to $R$ the root of the node-tree.

% Whereas $\delta_\mathbf{R}$ is unknown, $\tilde{\delta}_\mathbf{R}$ is always zero. The delay $\delta_N$ of node $N$ can be computed from $\tilde{\delta}_N$ of all pseudo-delays within the node-tree is known by subtracting the minimal $\tilde{\delta}$.
%\begin{equation}\label{eq:delaypseudo}
%  \delta_N = \tilde{\delta}_N - \min_{X \in \nset}{\tilde{\delta}_X}
%\end{equation}
The differences with respect to $R$ can be calculated by a depth-first search from root $R$ to all descendant nodes by adding the differences between each parent and its child:
\begin{equation}\label{eq:pseudodelay}
    {\delta}_Q = {\delta}_P + \text{PMW}(Q) - \text{PMW}(P), \hspace{.3cm} Q\leq P.
\end{equation}
where ${\delta}_N = \text{PMW}(N) - \text{PMW}(R)$. 

% \Cref{eq:pmwdif} can be easily explained through an example of the cluster in \Cref{fig:pmw}. The PMW's of $A, B, C$ are respectively 
% \begin{align}
%     \nonumber \text{PMW}(A) &= \rho_A + \abs{(B, C)}, \\
%     \nonumber \text{PMW}(B) &= \rho_B + \abs{(A, B)} + \abs{(B, C)}, \\
%     \nonumber \text{PMW}(C) &= \rho_C + \abs{(A, B)}.
% \end{align}
% Take $A\geq B\geq C$. The differences in node delay are thus 
% \begin{align}
%     \nonumber \Delta_B &= \delta_B - \delta_A =& 2\left(\rho_B - \rho_A + \abs{(A, B)}\right), \\
%     \nonumber \Delta_C &= \delta_C - \delta_B =& 2\left(\rho_C - \rho_B - \abs{(B, C)}\right).
% \end{align}

Equations \Cref{eq:pmwdif,eq:pseudodelay,eq:snodeparity,eq:jnodeparity} show that the differences in PMW with respect to the root node can be calculated through two depth-first searches (DFS) of $\nset$. In the first DFS, the \emph{parity DFS}, we calculate the node parity $\pi_N$ a via tail-recursive function using \Cref{eq:snodeparity,eq:jnodeparity}. In the second DFS, the \emph{delay DFS}, we calculate the PMW differences $\delta_N$ via a head-recursive function using \Cref{eq:pmwdif,eq:pseudodelay}. 

\textcolor{red}{MOVE? or expand, growth DFS is not yet introduced}During this second DFS, the minimal PMW difference within $\nset$ can be kept track of by comparison for calculating the node delay $\delta_N$ at a later instance during the \emph{growth DFS}. 

\subsection{Growing node-trees}\label{sec:grownodetrees}

A single growth iteration, which is applied in the Union-Find decoder by adding half-edges to all boundary vertices of the cluster, is now replaced by another DFS of $\nset$ from the same root $\mathbf{R}$ as before. During this DFS, $\delta_N$ of every node $N$ is calculated via \Cref{eq:delaypseudo} and $N$ is conditionally grown --- adding half-edges to the boundary vertices in the current node and adding $\nicefrac{1}{2}$ to its radius $\rho_N$ --- if $\delta_N = 0$. This requires us to save the list of boundary vertices to each node (\Cref{fig:nodetypes}c). When the delays $\delta_N$ for all nodes in $\nset$ are zero, all nodes are grown simultaneously within the same iteration. 

As the node radii increase, the values for $\delta_N$ across $\nset$ change. This would mean that the DFS's of node parity and delay need to be done after each growth iteration. Luckily, if no new nodes are added to $\nset$ after a growth iteration, which is the case if no mergers occur between clusters, $\pi_N$ on iteration $t+\theta$ can be calculated via previous values of pseudo-delays in iteration $t$ by the introduction of the node \textbf{wait} parameter $\omega_N$. During the growth DFS, if $\delta_N \neq 0$, add 1 to $\omega_N$. If no nodes are added to $\nset$, the node delay in the next iteration are
\begin{multline}\label{eq:delay}
    \delta_N(t+\theta) = \tilde{\delta}_N(t) - \min_{X \in \nset(t)}{\tilde{\delta}_X} - \omega_N(t+\theta-1),  \\
    \hspace{.3cm} N \in \nset, \hspace{.3cm} \nset(t+\theta) = \nset(t).  
\end{multline}

% Note that we have not stated which node in $\nset$ should be the root node. In fact, any node in $\nset$ could have been picked as the root of the node-tree. The only requirement is that the DFS of cluster growth must be performed along the same direction as the DFS's of the parity and delay calculations. If no cluster mergers occur, the node delays can be reused in the node suspension calculation prior to node growth. The node-tree is constructed by storing all neighbors of a node to a list. This way, the DFS's' direction can be determined by simply saving the root node, the starting point of the DFS's, to the cluster. All node variables are depicted in \Cref{fig:nodetypes}c. 
% In the next section, we expand upon this idea of "reusing" some intermediate parameters to calculate the node suspensions after a cluster merger.  

\subsection{Joining node-trees}\label{sec:nodejoin}

\Figure[b](topskip=0pt, botskip=0pt, midskip=0pt){figures/tikz/build/main-figure5.pdf}{
    \emph{(a)} An odd cluster $\nset_o=\{A, B, \mathbf{O}\}$ with root $A$ joins with an even cluster $\nset_e=\{C, \mathbf{E}\}$ with root $C$ on nodes $\mathbf{O}, \mathbf{E}$, respectively, to a joined node-tree. If we choose to \emph{(b)}, make $\mathbf{E}$ a child of $\mathbf{O}$, the parities and delays the subtree of $\nset_o$ can unchanged, and we only have to perform partial parity and delay calculations over the subtree of $\nset_e$. If we choose to \emph{(c)}, make $\mathbf{O}$ a child of $\mathbf{E}$, parities and delays have to be recalculated in the entire joined node-tree. \label{fig:inversion}}

\Figure[tb](topskip=0pt, botskip=0pt, midskip=0pt){figures/tikz/build/main-figure4.pdf}{
    The relevant data structures. \emph{(a)} The cluster-tree of the Union-Find data structure, whose elements are vertices $v$. The path from a vertex to the root of the cluster-tree is traversed to find the root element, which differentiates clusters. Next to the cluster size and parity, the root node $\mathbf{R}$ of the node-tree $\nset$ is stored at the root of the cluster-tree. Each vertex $v$ additionally stores to which node $N$ it belongs. \emph{(b)} A spanning-tree $\vset$ with 3 non-trivial vertices. As $\vset$ is strictly acyclic, the cluster's edges must be maintained such that no cycles are created. This is done during growth by removing edges (red dotted lines) if a cycle is detected. \emph{(c)} The node-tree $\nset$, which consists of the syndrome-nodes $\dot{A}, \dot{B}, \dot{C}$ with primer vertices $a, b, c$, and junction-nodes $\bar{D}, \bar{I}$ with roots $d,i$. At each node $N$ and the root $\mathbf{R}=\dot{B}$ various variables are stored.\label{fig:nodetypes}}

As clusters grow in size they merge with other clusters. 
%In the Union-Find (UF) algorithm, odd clusters
%parity clusters of an odd number of non-trivial vertices, which are elements of $\sigma$, 
%grow %in size 
%repeatedly and merge with other clusters until all clusters are even. 
%The data structures of the merged cluster can be obtained by merging the data structures of the original clusters: %During these mergers, the node-trees and the cluster-trees are merged. %of the Partitioned-Growth data structure are combined. 
%Let us first make a  distinction between the merging protocols of the underlying data structures; 
%the cluster-trees are merged with the $\Union$ function, whereas 
In the following we describe how to obtain the node-tree of the merged cluster from the original node-trees. % are merged with the $\Nodejoin$ function which we describe in the following. %After a join of multiple node-trees, the node delays within the combined node-tree change. Therefore, the $\Nodejoin$ protocol's focus is to minimize the DFS's of the recalculation of the node parity and delays in the combined node-tree. 

%Let the parity $\nset$ be the number of syndrome-nodes in $\nset$, which is equivalent to the parity of the cluster of $\nset$. 

Since only odd clusters grow, only two types of clusters can merge: even with odd and odd with odd. 

For even node-trees, the PMW is undefined since a perfect matching would not be possible when merging with a non-trivial vertex. Consequently, node parity and delays are undefined for even node-trees and do not need to be calculated. 

The second type of merger is between an even and an odd cluster. %The combined cluster is odd, and its growth is continued. Thus, its node delays must be computed. 
In order to define the new node-tree, we need to choose a new root. We consider two different Options, choosing the root of the odd cluster and choosing the root of the even cluster:

Let $\nset_o,\nset_e$ be the odd and even node-trees that are to be joined, and let $\mathbf{O}\in \nset_o,\mathbf{E} \in \nset_e$ be the nodes connected (\Cref{fig:inversion}\emph{a}).
If $\nset_o$'s root becomes the root of the joined node-tree (\Cref{fig:inversion}\emph{b}), $\mathbf{E}$ becomes a child node of $\mathbf{O}$. 
Since $\nset_e$ contains an even number of syndrome-nodes, the node parities in $\nset_o$ do not change. 
Hence, the node parity DFS is only necessary in the subtree $\nset_e$, which now has $\mathbf{E}$ as subroot. 
Furthermore, as the node delay depends only on its own properties and its parent's, the node delay DFS is also only required from node $\mathbf{E}$ and within the subtree of $\nset_e$. 
Hence, only partial DFS's within the even cluster are required. % of the node-tree are precisely what was required, as the node parity and delays in $\nset_e$ were undefined. 

Alternatively, if $\nset_e$'s root becomes the root of the combined tree (\Cref{fig:inversion}\emph{c}), an odd number of syndrome-nodes are attached to $\mathbf{E}$. In consequence the parities of the nodes on the path from $\mathbf{E}$ to the root are changed. This choice of root requires the DFS's on the entire combined node-tree to calculate for node parities and delays. 

For this reason, we choose to keep the root of the odd node-tree, which we dub \textbf{Odd-Rooted Join}. 

In addition, a cluster can be subjected to multiple mergers within the same growth iteration, during which the parity of the merged cluster changes depending on the number of mergers and the parities of the clusters involved. The DFS's related to the parity and delay calculations must, for this reason, not be initiated directly after the joining of node-trees. After all, it may be possible for the cluster to merge again such that the parities and delays become invalid. To prevent these redundant calculations, subroots of the even subtrees are stored to a list $\m{R}$ at the root of the node-tree (\Cref{fig:nodetypes}\emph{c}). When multiple mergers occur, the root node that stores the now redundant subroots is replaced by a new root with new $\m{R}$. If a cluster is selected for growth, we bar for the subroots in $\m{R}$ at the new root node and initiate the DFS's from these subroots. We call this the \textbf{Root List Replacement}. 


\subsection{Parity Inversion}\label{sec:inversion}

\Figure[tb](topskip=0pt, botskip=0pt, midskip=0pt){figures/tikz/build/main-figure6.pdf}{
    The node delays $\delta_N$ for nodes for 3 odd node-trees $\{\nset_1, \nset_2, \nset_3\}$ of 3 nodes that grow and join into a single node-tree. \emph{(a)} Node delays are calculated via \Cref{eq:pmwdif,eq:pseudodelay,eq:delaypseudo}. In step 1, the growth in each of the three node-trees' outer nodes is prioritized, and the node-trees merge. In step 2, the recalculation of the joined node-tree is performed. Parities within the subtree of $\nset_2$ are now inverted, and the  in these nodes have doubled. \emph{(b)} Node delays are calculated with \Cref{eq:pmwdifnew} instead of \Cref{eq:pmwdif}. Now the increase in $\delta_N$ after parity inversion is halved.\label{fig:partialdfs}}

An unfortunate effect, which we dub \textbf{parity inversion}, causes a decrease in the algorithm's performance as the lattice size is increased. We will demonstrate this effect through the example in \Cref{fig:partialdfs}\emph{a}. Consider three instances of the node-tree of \Cref{fig0}; $\nset_a, \nset_b, \nset_c$, positioned near each other on the lattice. For each node-tree, if the middle node is suspended from growth for two iterations, all nodes have the same Potential Matching Weight. However, in the current example, the node-trees $\nset_a, \nset_b, \nset_c$ merge after 1 iteration. The combined node-tree is odd. Thus, we recalculate the node parities and delays to find that the parities in the partition of the node-tree containing the nodes of $\nset_b$ have been inverted, and the node suspensions in this partition have doubled with respect to their value before the merger. If the next merging event occurs on the node with the doubled node suspension, the matching weight may be large, in contrast with the goal of decreasing the matching weight. Moreover, the effect of parity inversion accumulates if subsequent mergers occur before zero node suspension has been reached. %Nevertheless, as more inversions occur, the maximum node suspension in the node-tree increases, and it becomes more and more unlikely for a cluster to actually reach zero node suspension in all nodes. The number of inversions is directly related to the number of merging events, and thus to the size of the lattice. The performance to improve the heuristic for minimum weight matching thus decreases for larger lattices. 

%Parity inversion defines a trade-off in the Partitioned-Growth data structure: `A node must wait as many iterations as it is suspended to reach equilibrium in Potential Matching Weight in the node-tree. However, after Parity Inversion, the node suspension for previously prioritized nodes increases linearly with the number of iterations waited by the suspended nodes pre-inversion.' As a compromise, 
To compensate the effect of parity inversion, we redefine the node delay as \textbf{half} the number of growth iterations needed for all nodes in the node-tree to reach equal PMW. This changes \Cref{eq:delayequation,eq:pmwdif} to:
\begin{align}
    \delta_N &= \text{PMW}(N) - \min_{X\in\nset}{\text{PMW}(X)}, & N \in \nset, \tag{\ref{eq:delayequation}'}  \\
    \delta_Q - \delta_P &= \rho_Q-\rho_P - (-1)^{\pi_Q}\abs{(P,Q)}, & Q\leq P. \tag{\ref{eq:pmwdif}'}\label{eq:pmwdifnew}
\end{align}

\subsection{Algorithm description}\label{sec:pseudocode}

The full version of the algorithm we have described is given in Algorithm \ref{algo:ufbb}. Note that this pseudocode includes instructions that are shortened versions of the pseudocode of the Union-Find decoder \cite{delfosse2017almost}. This is done for clarity on the modified additions of the Partitioned-Growth data structure and protocols on top of the Union-Find pseudocode. The first block of lines \ref{algo:B1a}-\ref{algo:B1b} initializes the clusters and describes the loop of cluster growth. Block 2 contains lines \ref{algo:B2a}-\ref{algo:grow} and describes the DFS's related to calculating the node parities and delays from all even subroots stored at the root node, and the DFS of the cluster growth. Block 3 contains lines \ref{algo:B3a}-\ref{algo:B3b} and describes the combined merging protocols of the Union-Find and Partitioned-Growth data structures. Note that lines \ref{algo:dfa}-\ref{algo:dfb} contain an extra step that maintains acyclic spanning-trees. The final block in line \ref{algo:B4a} is the peeling decoder \cite{delfosse2017linear}, which now does not have to create the spanning forest of the grown clusters. Similarly to the Union-Find decoder, Weighted Growth is applied such that the smaller cluster is always grown first. 

\begin{algorithm}[h!]
  \BlankLine
  \KwData{A graph $G=(V,E)$, and syndrome $\sigma \subseteq V$}
  \KwResult{Correction set $\m{C}$}
  \BlankLine
  Initialize cluster spanning-trees and node-trees.\;\label{algo:B1a}% and table $\Support$
  Create the list $\m{L}$ of odd clusters.\;
  \While(){$\m{L}$ is not empty}{
    Initialize the fusion list $\m{F}$ as an empty list.\;\label{algo:B1b}
    \For(){cluster $\in\m{L}$ \label{algo:B2a}}{
      \For(){$N \in\m{R}$ at root node $\mathbf{R}$ of $\nset$}{
        Apply DFS's to calculate node parities and delays (\Cref{eq:pseudodelay,eq:snodeparity,eq:jnodeparity}) from $N$ to all descendant nodes. Keep track of $\min_{X \in \nset}{\tilde{\delta}_X}$ during the delay DFS.\;\label{algo:pdc}
      }
      Apply DFS from root $\mathbf{R}$ to all descendants. At each node $N$ during the DFS, if $\delta_N=0$ (\Cref{eq:delay}), grow all boundary edges of vertices in the node a half-edge per the Union-Find decoder, such that grown edges are added to $\m{F}$, and apply $\rho_N=\rho_N +\nicefrac{1}{2}$. \;\label{algo:grow}
      %If $\m{s}(n)\neq0$, apply $n_w=n_w+1$ and continue the DFS.
    }
    \For(){edge $(n,m) \in \m{F}$\label{algo:B3a}}{
      \eIf(){$\Find(n)\neq\Find(m)$}{
        Merge cluster-trees by Weighted $\Union$.\;
        \eIf(){$n \in N$ and $m \in M$\label{algo:joina}}{
          Merge node-trees by Odd-Rooted $\Nodejoin$. If resulting node-tree with root $\mathbf{R}$ is odd, add even subroot (either $N$ or $M$) to list $\m{R}$ at $\mathbf{R}$.\;
        }($n$ or $m$ does not belong to a cluster){
          Add $n$ to $N$ or $m$ to $M$.\;\label{algo:joinb}
        }
      }($n,m$ in same cluster.\label{algo:dfa}){
        %Subtract 1 from $(u,v)$ in $\Support$.\;
        Nothing to keep both $\vset$ and $\nset$ acyclic.\;\label{algo:dfb} 
      }
    }
    Update $\m{L}$ with odd clusters\; \label{algo:B3b}
  }
  Apply the peeling decoder \cite{delfosse2017linear}.\label{algo:B4a}
  \caption{Union-Find Partitioned-Growth}\label{algo:ufbb}
\end{algorithm}