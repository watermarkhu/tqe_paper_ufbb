
\section{Complexity of Partitioned-Growth}\label{sec:complexity}

In this section, we will find the worst-case time complexity of the Union-Find Partitioned-Growth decoder. The additional cost with respect to the original Union-Find decoder can be split in two parts: (A) the depth-first-searches (DFS's) related to the (re)calculation of the node parities and node delays in line \ref{algo:pdc}, and (B) the DFS related to the growth of a cluster in line \ref{algo:grow}. We dub the two parts the \textbf{delay cost} and the \textbf{growth cost}, respectively. The $\Nodejoin$ operation in lines \ref{algo:joina}-\ref{algo:joinb} only has a linear addition to the cost.

\subsection{Delay cost}\label{sec:suscomplexity}
The cost of each step the DFS (\eqref{}, \eqref{} and \eqref{}) is independent of the tree size. In consequence the delay cost is proportional to the total number of nodes encountered in the DFS'. In the following we bound this number, which we denote by $N_\delta$: 

%As a result of \emph{Odd-Rooted Join} and \emph{Root List Replacement}, 
$N_\delta$ is the sum of all even node-trees that join with another to an odd node-tree and grows afterwards. Even node-trees that join with multiple node-trees within the same growth iteration and the final set of even node-trees whose clusters are peeled thus do not count towards $N_\delta$. Let an odd node-tree be denoted by $\oset$, an even node-tree by $\eset$, and an even node-tree that counts towards the delay cost by $\bm{\eset}$. To find the worst-case time complexity, we analyze cluster growth by a time-reversed approach; starting from a single cluster at the end of growth, and move back in time to find its \emph{node-tree decomposition}.


\begin{definition}\label{def:fragmentation}
  
  Let the \textbf{fragmentation} $\frag_\oset$ of some odd node-tree $\pr{k}\oset$ return its two most recently joined predecessor node-trees, which are $\pr{k-1}\oset$ and $\pr{k-1}\eset$
  \begin{equation}\label{eq:pfo}
    \frag_\oset(\{\pr{k}\oset\}) = \{\pr{k-1}\oset, \pr{k-1}\eset \},
  \end{equation}
  where the prefix $k$ indicate the \textbf{generation} to which the node-tree belongs. Let $\frag_\eset$ fragment an even node-tree $\pr{k}\eset$ to some even number $\nu$ of its predecessor odd node-trees  
  \begin{equation}\label{eq:pfe}
    \frag_\eset(\{\pr{k}\eset\}) =\{\pr{k}\oset_1,...,\pr{k}\oset_{\nu} \},
  \end{equation}
  which have joined to $\pr{k}\eset$ within the same growth iteration. As an even node-tree does not grow, it belongs to the same generation as its odd predecessors. Since $\frag_\oset$ and $\frag_\eset$ operate on node-trees of different parties, both can be combined as a \textbf{fragmentation step} $\frag$, which takes an odd node-tree $\pr{k}\oset$ and returns its $\nu+1$ predecessors
  \begin{multline}\label{eq:fstep}
    \frag(\{\pr{k}\oset\}) = \frag_\eset(\frag_\oset(\{\pr{k}\oset\})) =  \\ \{\pr{k-1}\oset_0,\pr{k-1}\oset_1,...,\pr{k-1}\oset_\nu\}, 
  \end{multline}
  where $\pr{k-1}\oset$ from \Cref{eq:pfo} is now labelled as $\pr{k-1}\oset_0$.
\end{definition}

Note that node-trees of the same \emph{generation} might not have been constructed in the same growth \emph{iteration}. An odd cluster grow for many iterations, while its node-tree does not change, before it merges with another node-tree of the same generation. We use the notation $\frag^{(2)}(\pr{k}\oset)$ to indicate that two fragmentation steps are applied on $\pr{k}\oset$ to obtain its predecessors of generation $k-2$. Furthermore, let $\frag_\oset^{(i)}(\nset)$ and be equivalent to $\frag_\oset(\frag^{(i-1)}(\nset))$. 

\Figure[hbt](topskip=0pt, botskip=0pt, midskip=0pt){figures/tikz/build/main-figure7.pdf}{
  The \emph{fragmentation} of node-tree $\pr{k+1}\oset$ into its predecessor node-trees, where the prefix $k+1$ indicates its \emph{generation}. The fragmentation step $\frag(\pr{k+1}\oset)$ returns $\nu+1$ odd predecessors of generation $k$. Fragmentation step $\frag$ can be separated into $\frag_\oset(\pr{k+1}\oset)$ that returns an odd predecessor $\pr{k}\oset_0$ and even predecessor $\pr{k}\eset$, and subsequently the $\frag_\eset(\pr{k}\eset)$, which returns $\nu$ odd predecessors of generation $k$. The figure depicts a fragmentation step with $\nu=2$.\label{fig6}}

\begin{definition}\label{def:decomposition}
  Let the \textbf{decomposition} of a node-tree $\nset$ be a series of $\mu$ fragmentation steps on $\nset$, such that the output of $\frag^{(\mu)}$ is a set of node-trees that have no predecessors. 
\end{definition}

Using node-tree decomposition, we aim to find the fragmentations that maximizes the number and size of $\bm{\eset}$ node-trees. The definitions of $\frag_\oset$ and $\frag_\eset$ matches the rules of \emph{Odd-Rooted Join} and \emph{Root List Replacement} such that each of the even node-trees in the output of $\frag_\oset$ is a $\bm{\eset}$ node-tree. The maximum value for $N_\delta$ can thus be computed using the decomposition of a maximally large odd node-tree $\pr{\mu}\oset$. 
\begin{equation}\label{eq:npdc}
  N_\delta = 2\sum_{k=1}^\mu{ \sum_{ \pr{\mu-k}\eset \in \frag_\oset^{(k)}(\pr{\mu}\oset) }{ \abs{\pr{\mu-k}\eset}} }.
\end{equation}
The inner sum accounts for the combined size of all even node-trees during a single fragmentation step $\frag_\oset^{(k)}$. The outer sum loops over all $\mu$ fragmentation steps. Finally, the $2$ refers to the individual DFS's of the parity and delay calculations. Note that in the decomposition of $\pr{\mu}\oset$, the lowest generation of node-trees outputted by $\frag^{(\mu)}(\pr{\mu}\oset)$ is $1$. 


We will use two simplifications to \Cref{eq:npdc} to find $N_\delta$. Junction-nodes are initiated on the tangent of two node radii belonging to separate node-trees when merging into one. With increasing number of fragmentations, the total number of nodes in the fragmented set must therefore decrease. We thus change \Cref{eq:npdc} into the inequality
\begin{equation}\label{eq:npdc2}
  N_\delta \leq 2\sum_{k=1}^\mu{ \sum_{ \pr{\mu-k}\eset \in \frag_\oset^{(k)}(\pr{\mu}\oset) }{ \abs{\pr{\mu-k}\eset}}}. 
\end{equation}

The existence of junction-nodes will be ignored in the remainder of this section, which does not compromise the inequality. If only syndrome-nodes exist, $\pr{k}\oset$'s size must equal the sum of sizes of its predecessors in $\frag(\pr{k}\oset)$. The size of each of the $\nu+1$ predecessors $\pr{k}\oset_i$ can be represented by the \textbf{fragmentation ratio} $\lambda$
\begin{equation}\label{eq:ratio}
  \pr{k}\lambda_i = \frac{\abs{\pr{k-1}\oset_i}}{\abs{\pr{k}\oset}}, \hspace{0.5cm} \sum_{i=0}^{\nu}{\pr{k}\lambda_i} = 1.
\end{equation}
Secondly, we assume that vertex-trees do not increase in size, such that $\abs{\nset}=\abs{\vset}$. Normally, the number of nodes in a cluster is bounded by the number of vertices $\abs{\nset}\leq \abs{\vset}$, as non-trivial vertices can be added to the node, which increases the node radius. By this assumption, the vertex-tree can only increase in size due to a merger between clusters, and nodes are effectively not allowed to increase in radius. While this is not possible during realistic cluster growth, this assumption allows us to further simplify \Cref{eq:npdc2}. 

To find the upper bound in $N_\delta$, we are now tasked to find: (a) $\nu$, the number of predecessors in $\frag_\eset$, (b) the fragmentation ratios $\{\lambda_0, ..., \lambda_\nu\}$, (c) the number of fragmentation generations $\mu$, and (d) the size of $\pr{\mu}\oset$. 

\begin{lemma}\label{lem:evenconstant}
  For constant fragmentation ratios $\pr{k}\lambda_i = \lambda_i$ during all generations $1\leq k<\mu$, the inner sum of \Cref{eq:npdc} is constant:
  \begin{equation*}
    \sum_{ \pr{\mu-k}\eset \in \frag_\oset^{(k)}(\pr{\mu}\oset) }{ \abs{\pr{\mu-k}\eset}} = C \hspace{1em}\forall k.
  \end{equation*}
\end{lemma}
\begin{proof}
  For $k=1$, there is an even-parity predecessor $\pr{\mu-1}\eset$ of size 
  \begin{equation*}
    \abs{\pr{\mu-1}\eset} = (1 - \lambda_0)\abs{\pr{\mu}\oset}.
  \end{equation*}
  For $k=2$, every odd node-tree in $\{\pr{\mu-1}\oset_0,... ,\pr{\mu-1}\oset_\nu\}$ is fragmented by $\frag_\oset$ to an even predecessor $\pr{\mu-2}\eset_i$ for $i \in \{0,...,\nu \}$, such that 
  \begin{equation*}
    \sum_{i=0}^{\nu}{\abs{\pr{\mu-2}\eset_i}}  = \sum_{i=0}^{\nu}{\lambda_i(1 - \lambda_0)\abs{\pr{\mu}\oset}}= (1 - \lambda_0)\abs{\pr{\mu}\oset}.
  \end{equation*}
  The same is true for all generations $1\leq k<\mu$. 
\end{proof}

\begin{theorem}\label{the:fragnumber}
  The upper bound of $N_\delta$ is obtained for $\nu=2$, such that $\frag_\eset$ of even node-tree $\pr{k}\eset$ returns two odd predecessors. 
\end{theorem}
\begin{proof}
  The sum of even node-tree sizes in every generation is constant per \Cref{lem:evenconstant}. Thus, the upper bound in \Cref{eq:npdc2} is obtained by the largest possible $\mu$. As $\nu$ increases the number of odd node-trees in each $\frag^{(k)}_\oset$, the average size of these odd node-trees decreases. Since the size of a node-tree is proportional to the number of predecessor generations, we find that 
  \begin{equation*}
    \mu \propto \frac{1}{\nu}. 
  \end{equation*}
  Hence, the upper bound in \Cref{eq:npdc2} exists in the minimal value of $\nu$, which is $\nu = 2$.
\end{proof}

Using \Cref{the:fragnumber}, a fragmentation step on an odd cluster $\frag(\pr{k+1}\oset)$ now returns $\{\pr{k}\oset_0, \pr{k}\oset_1, \pr{k}\oset_2\}$, where $\pr{k}\oset_1, \pr{k}\oset_2$ are predecessors of the even node-tree $\pr{k}\eset$. 

\begin{lemma}\label{lem:chrono}
  The node-tree size of $\pr{k}\oset_0$ must be smaller than or equal to $\pr{k}\oset_1, \pr{k}\oset_2$, such that $\lambda_1 \geq \lambda_0 \leq \lambda_2$. 
\end{lemma}
\begin{proof}
  The partial fragmentations must occur in the order of first \Cref{eq:pfo}, then \eqref{eq:pfe}, as \eqref{eq:pfe} requires an even node-tree that is returned by \eqref{eq:pfo}. In terms of cluster growth, the vertex-trees $\vset_1, \vset_2$, corresponding to $\pr{k}\oset_1, \pr{k}\oset_2$, must merge before the combined vertex-tree can merge with $\vset_0$, which corresponds to $\pr{k}\nset_0$. As a result of Weighted Growth, $\abs{\vset_1}$ and $\abs{\vset_2}$ must be smaller or equal to $\abs{\vset_0}$, such that 
  \begin{equation*}
    \abs{\vset_1}\geq \vset_0 \leq\abs{\vset_2}.
  \end{equation*}
  If this condition is not met, the cluster of $\vset_0$ grows first and merges with either $\vset_1$ or $\vset_2$, and the causality of events is disturbed. Since we assumed $\abs{\nset}=\abs{\vset}$, this can be translated to 
  \begin{equation*}
    \abs{\nset_1}\geq \nset_0 \leq\abs{\nset_2},
  \end{equation*}
  and subsequently to the fragmentation ratios.
\end{proof}

\begin{theorem}\label{the:ratios}
  The upper bound for $N_\delta$ is obtained via the fragmentation ratios $\lambda_0 = \lambda_1 = \lambda_2 = \nicefrac{1}{3}$.
\end{theorem}
\begin{proof}
  The ratios $\{\lambda_0, \lambda_1, \lambda_2\}$ can be found by maximizing the size of the even node-tree $\pr{k}\eset$ in each fragmentation step, which is 
  \begin{equation*}
    \abs{\pr{k}\eset} = (\lambda_1 + \lambda_2)\abs{\pr{k-1}\oset}.
  \end{equation*}
  Since $ \lambda_1 \geq \lambda_0 \leq \lambda_2$ per \Cref{lem:chrono}, the largest values for $\lambda_1, \lambda_2$ possible are equal to $\lambda_0$.
\end{proof}

The last unknown parameters in finding the upper bound of $N_\delta$ in are $\mu$ and $\abs{\pr{\mu}\oset}$.

\begin{theorem}\label{the:km}
  For $\nu = 2$ and $\lambda_i = \{\nicefrac{1}{3},\nicefrac{1}{3},\nicefrac{1}{3}\}$, the maximum number of fragmentation generations is $\mu = \log_3{\abs{\pr{\mu}\oset}}$.
\end{theorem}
\begin{proof}
  In every generation, all node-trees are fragmented into 3 predecessors that are $\nicefrac{1}{3}$ the size of their common successor. The series of $\mu$ fragmentation steps is thus simply $\mu$ divisions of the node-tree $\pr{\mu}\oset$ in 3 parts until all predecessors have size 1, at which point a node-tree cannot be fragmented.
\end{proof}

The maximum size of the odd node-tree $\pr{\mu}\oset$ is bounded by the system size $n$, the number of qubits on the lattice. Collecting \Cref{the:fragnumber,the:ratios,the:km} and filling in \Cref{eq:npdc2}, we find that

\begin{align*}
  \nonumber N_\delta &\leq 2\sum_{k=1}^\mu{ \sum_{ \pr{\mu-k}\eset \in \frag_\oset^{(k)}(\pr{\mu}\oset) }{ \abs{\pr{\mu-k}\eset}}  } \\
  \nonumber         &\leq 2\sum_{k=1}^{\log_3{\abs{\pr{\mu}\oset}}} \frac{2}{3}\abs{\pr{\mu}\oset}\\
                    &\leq \frac{4}{3}\abs{\pr{\mu}\oset}\log_3{\abs{\pr{\mu}\oset}} \\
                    &\leq \frac{4}{3}n \log_3{n}.
\end{align*}

The worst-case time complexity of related to the delay cost is thus $\Omega(n\log{n})$. 

\subsection{Growth cost}\label{sec:growthcost}

To grow a cluster represented by a node-tree $\nset$, a depth-first search (DFS) is performed on the node-tree to find all nodes with zero delay. The total cost of these DFS's is proportional to the total number of nodes encountered during these DFS's, which we dub $N_g$. Using \Cref{def:fragmentation,def:decomposition}, the cost of growth of a node-tree $\pr{\mu}\oset$ is proportional to the sum of sizes of all odd node-trees in the decomposition of $\pr{\mu}\oset$: 
\begin{equation}\label{eq:ngrow}
  N_g = 2\sum_{k=1}^\mu{ \sum_{ \pr{\mu-k}\oset \in \frag^{(k)}(\pr{\mu}\oset) }{ \abs{\pr{\mu-k}\oset}} }.
\end{equation}
Again, we assume that no trivial vertices are added to a cluster or $|\nset| = |\vset|$ such that \Cref{eq:ngrow} becomes an upper bound. As a result of \emph{Odd-Rooted Join} and \emph{Root List Replacement}, the upper bound is obtained in the largest $\mu$. This is again achieved through $\nu = 2$. For every fragmentation of some odd node-tree $\pr{k-1}\oset$ into $\{\pr{k}\oset_0, \pr{k}\oset_1, \pr{k}\oset_2\}$, all three predecessors add to $N_g$ if they have grown. As a result of Weighted Growth, this is the case when $\abs{\vset_0}\approx \abs{\vset_1}\approx\abs{\vset_2}$ such that $\lambda_0 \approx \lambda_1 \approx \lambda_2\approx \nicefrac{1}{3}$. For these values of $\nu$ and $\lambda$, we can apply \Cref{the:km} for $\mu$. For $|\nset| = |\vset|$, the sum of even node-tree sizes in every $\frag_\oset^{(k)}$ is exactly $\abs{\pr{\mu}\oset}$, and we find that
\begin{align*}
  \nonumber N_g &\leq 2\sum_{k=1}^\mu{ \sum_{ \pr{\mu-k}\oset \in \frag^{(k)}(\pr{\mu}\oset) }{ \abs{\pr{\mu-k}\oset}}  } \\
  \nonumber         &\leq 2\sum_{k=1}^{\log_3{\abs{\pr{\mu}\oset}}} \abs{\pr{\mu}\oset}\\
                    &\leq 2\abs{\pr{\mu}\oset}\log_3{\abs{\pr{\mu}\oset}}\\
                    &\leq 2 n \log_3{n},
\end{align*}
which again corresponds to a worst-case time complexity $\Omega(n\log{n})$.